\documentclass[a4paper,12pt]{book}
\usepackage{rotating}
\usepackage[english]{babel}
\usepackage[tikz]{mdframed}
\usepackage{xcolor}
\usepackage{tkz-tab}
\usepackage{xpatch}
\usepackage{enumitem}
%..................usetikzlibrary......................................
\usetikzlibrary{shapes.multipart}
\usetikzlibrary{decorations.pathmorphing}
\usetikzlibrary{decorations.footprints}
\usetikzlibrary{decorations.fractals}
\usetikzlibrary{positioning,shapes.multipart,shapes.callouts}
%%%%%%%%%%definecolor.....-------------*****
\definecolor{activityBackgroundColor}{RGB}{254,254,206}
\definecolor{activityBorderColor}{RGB}{168,0,54}
\definecolor{noteBackgroundColor}{RGB}{251,251,119}
\definecolor{noteBorderColor}{RGB}{168,0,54}
\definecolor{tcbcolback}{RGB}{255,128,0} %ពណ៌
%------------------ តម្រឹមក្រដាស់------------
\usepackage[top=1.5cm,bottom=2.4cm,right=1.5cm,left=1.5cm]{geometry}
%%%%%%%%%%%%%%%%%%ចុះបន្ទាត់
\XeTeXlinebreaklocale ''km''
\XeTeXlinebreakskip = 0pt plus 0.5pt minus 0.25pt
%%%%%%%%%%%%%%%%%%%%%
\everymath{\displaystyle}
\renewcommand{\baselinestretch}{1.8}
\usepackage{amsmath,amssymb}
\usepackage{polynom,xlop}
\usepackage{marvosym}
\usepackage[export]{adjustbox}
\usepackage[most]{tcolorbox} 
\usepackage{tikz}% graphic drawing
%---------------------------- ផ្លាស់ប្តូរលេខក្នុង enumerate ------
\usepackage{enumitem}% change list environment like enumerate, itemize and description
 \SetEnumitemKey{I}{label=\Roman*.}
%%
\SetEnumitemKey{1}{label=\color{red}\knum*., parsep=0pt,font=\kbk}
%%
\SetEnumitemKey{a}{label=\color{blue}\kalph*., parsep=0pt,font=\kbk}
\SetEnumitemKey{A}{label=\color{blue}\en\Alph*,parsep=0pt}
%អក្សរខ្មែរមានរង្វង់%%%%%%%%%%%%%
\SetEnumitemKey{ac}{%
	leftmargin=*,%
	label={\protect\tikz[baseline=-0.9ex]\protect\node[draw=gray,thick,circle,minimum height=.5cm,inner sep=1pt,text=white,fill=orange]{\kb\kalph*};},%
	font=\small\sffamily\bfseries,%
	labelsep=1ex,%
	topsep=0pt}
%លេខ មានរង្វង់  {I,II,II,IV,V,...}
\SetEnumitemKey{IC}{%
	leftmargin=*,
	label={\protect\tikz[baseline=-0.9ex]\protect\node[draw=gray,thick,circle,minimum height=.65cm,inner sep=1pt,text=white,fill=magenta]{\en\Roman*};},%
	font=\small\sffamily\bfseries,%
	labelsep=1ex,%
	topsep=0pt}
	%%លេខខ្មែរ មានរង្វង់(១,២,៣,៤,៥,៦,...)
\SetEnumitemKey{1c}{leftmargin=*,%
	label={\protect\tikz[baseline=-0.9ex]\protect\node[draw=gray,thick,circle,minimum height=.3cm,inner sep=1pt,text=black,fill=red]{\kb \knum*};},%
	font=\small\sffamily\bfseries,%
	labelsep=1ex,%
	topsep=0pt}
%%%លេខarabicមានរង្វង់
	\SetEnumitemKey{kc}{leftmargin=*,%
	label={\protect\tikz[baseline=-0.9ex]\protect\node[draw=gray,thick,circle,minimum height=.5cm,inner sep=1pt,text=white,fill=cyan]{\arabic*};},%
	font=\small\sffamily\bfseries,%
	labelsep=1ex,%
	topsep=0pt}	
%%%អក្សរអង់គ្លេស់តូចមានរង្វង់
	\SetEnumitemKey{ae}{leftmargin=*,%
	label={\protect\tikz[baseline=-0.9ex]\protect\node[draw=gray,thick,circle,minimum height=.5cm,inner sep=1pt,text=red,fill=green!50!]{\en\alph*};},%
	font=\small\sffamily\bfseries,%
	labelsep=1ex,%
	topsep=0pt}	
	%%%អក្សរអង់គ្លេស់ធំមានរង្វង់
	\SetEnumitemKey{Ae}{leftmargin=*,%
	label={\protect\tikz[baseline=-0.9ex]\protect\node[draw=blue,thick,circle,minimum height=.6cm,inner sep=1pt,text=blue,fill=white]{\en\Alph*};},%
	font=\small\sffamily\bfseries,%
	labelsep=1ex,%
	topsep=0pt}	
\usepackage{multicol}% multi columns
%---------------------- ដាក់ font khmer--------
\usepackage[no-math]{fontspec}
\usetikzlibrary{shapes.callouts,shadows.blur,positioning,arrows}
\usepackage{amssymb}
\usepackage{wasysym}
\setmainfont [Scale = 0.84, Script=Khmer]{Khmer OS Battambang}
\setmathrm{Times New Roman}
\newcommand{\ko}{\fontspec[Scale = 0.84, Script=Khmer]{Khmer OS}\selectfont}
\newcommand{\kml}{\fontspec[Scale = 0.84, Script=Khmer]{Khmer OS Muol Light}\selectfont}
\newcommand{\kos}{\fontspec[Scale = 0.84, Script=Khmer]{Khmer OS System}\selectfont}
\newcommand{\kbk}{\fontspec[Scale = 0.84, Script=Khmer]{Khmer OS Bokor}\selectfont}
\newcommand{\kb}{\fontspec[Scale = 0.84, Script=Khmer]{Khmer OS Battambang}\selectfont}
\newcommand{\kf}{\fontspec[Scale = 0.84, Script=Khmer]{Khmer OS Fasthand}\selectfont}
\newcommand{\kom}{\fontspec[Scale = 0.84, Script=Khmer]{Khmer OS Muol Pali}\selectfont}
\newcommand{\kr}{\fontspec[Scale=0.84,Script=Khmer]{Kh Rayuth HD 1}\selectfont}
\newcommand{\ka}{\fontspec[Scale=0.84,Script=Khmer]{Khmer Arrow}\selectfont}
 \newcommand{\kv}{\fontspec[Scale = 0.84, Script=Khmer]{Khmer Viravuth}\selectfont}
\newcommand{\en}{\fontspec{Times New Roman}\selectfont}
%%%%%--------------លេខខ្មែរ------------------
\makeatletter
\def\@khmernum#1{\expandafter\@@khmernum\number#1\@nil} 
\def\@@khmernum#1{%
\ifx#1\@nil 
\else 
\char\numexpr#1+"17E0\relax 
\expandafter\@@khmernum\fi
} 
\def\knum#1{\expandafter\@khmernum\csname c@#1\endcsname}
\def\khmernumeral#1{\@@khmernum#1\@nil}
\AddEnumerateCounter{\knum}{\@knum}{}
\makeatother
%----------------------អក្សរខ្មែរ---------
\makeatletter
\newcommand*{\kalph}[1]{%
\expandafter\@kalph\csname c@#1\endcsname%
}
\newcommand*{\@kalph}[1]{%
\ifcase#1\or ក\or ខ\or គ\or ឃ\or ង\or ច\or ឆ\or ជ\or ឈ\or ញ\or  ដ\or ឋ\or ឌ%
\or ឍ\or ណ\or ត\or ថ\or ទ\or ធ\or ន\or ប\or ផ\or ព\or ភ\or ម\or យ\or រ\or ល%
\or វ\or ស\or ហ\or ឡ\or អ%
\else\@ctrerr\fi%
}
\AddEnumerateCounter{\kalph}{\@kalph}{}
\makeatother
%----------------------------------------
\usepackage{fancyhdr}
\newcommand*\olive[1]{\tikz[baseline=(char.base)]{
            \node[shape=circle,fill=white,draw,inner sep=3pt] (char) {#1};}}
% header style
\pagestyle{fancy}
\fancyhf{}
\fancyhead[R]{\begin{tikzpicture}
	\draw [black,line width=0.7mm](0,0)--(3,0);
\end{tikzpicture}
\kml ​ថា​ មករា}
\fancyhead[L]{\includegraphics[scale=0.3]{vogeltje.png}  }
\fancyfoot[LE,RO]{\color{blue}\olive{\color{blue}\thepage}}
\fancypagestyle{plain}{%
  \fancyhf{}
  \fancyfoot[LE,RO]{\color{blue}\olive{\color{blue}\thepage}}
  \renewcommand{\headrulewidth}{0pt}
}
\fancyfoot[RE,LO]{}
\renewcommand{\thepage}{\kml \knum{page}}
%%%%%%
\futurelet\TMPheadrule\def\headrule{{\color{white}\TMPheadrule}}
\headheight = 20pt
\headsep = 5pt
\footskip = 25pt
%%%%%renewcommand
\renewcommand{\thechapter}{\kml \knum{chapter}}
\renewcommand{\thesection}{\kml \knum{section}} 
\renewcommand{\thesubsection}{\thesection.\knum{subsection}}
\renewcommand{\thesubsubsection}{\thesubsection.\knum{subsubsection}}
\renewcommand{\thepart}{\kml \knum{part}}
%%%%%%%%newcommand
\newcommand{\N}{\mathbb{N}}
\newcommand{\R}{\mathbb{R}}
\newcommand{\I}{\overrightarrow{i}}
\newcommand{\J}{\overrightarrow{j}}
\newcommand{\K}{\overrightarrow{k}}
\newcommand{\Z}{\mathbb{Z}}
\newcommand{\so}{{\color{magenta}{\kv ដូចនេះ}}\quad}
\newcommand{\answer}{\tcbox[enhanced,size=fbox,colback=gray!31,colframe=gray!1,drop lifted shadow]}
\newcommand{\solution}{{\color{magenta}{\underline{\kv ដំណោះស្រាយ}}}}
%%%%%%%%%%%%ប្រអប់លំហាត់
\tcbset{theostyle/.style={
    enhanced,
    rounded corners,lifted shadow={1mm}{-2mm}{3mm}{0.1mm}%
{black!50!white},
    attach boxed title to top left={
      xshift=-1mm,
      yshift=-4mm,
      yshifttext=-1mm
    },
    top=1.5ex,
    colback=red!10!white,
    colframe=white,
    fonttitle=\bfseries,
    boxed title style={
      rounded corners,
    size=small,
    colback=red,
    colframe=white,
    lifted shadow={1mm}{-2mm}{3mm}{0.1mm}%
{black!50!white}
  } 
}}
\newtcbtheorem{Theorem}{\kv ទ្រឹស្តីបទទី}{%
  theostyle
}{theo}
\newenvironment{theorem}{\Theorem{}{}}{\endTheorem}
%%%%%%\proof
\usepackage[utf8]{inputenc}
\usepackage[english]{babel}
\usepackage{amsmath,amsthm,amssymb}
\usepackage{tcolorbox}
\tcbuselibrary{skins}
\tcbuselibrary{breakable}
\tcolorboxenvironment{proof}{% 'proof' from 'amsthm'
    blanker,breakable,left=5mm,
    before skip=10pt,after skip=10pt,
    borderline west={1mm}{0pt}{magenta}
}
%------------solution box
\newtcolorbox{solutionbox}{
        colframe=cyan!20!white,
        colback =cyan!20!white,
        top=0mm, bottom=0mm, left=0mm, right=0mm,
        arc=0mm,
%
        fontupper=\color{blue!70!black},
        fonttitle=\bfseries\color{blue!70!black},
        title=គន្លឹះដោះស្រាយ:
                        }
 %******--------marker note
\newtcolorbox{marker}[1][]{enhanced,
  before skip=2mm,after skip=3mm,fontupper=\kbk,
  boxrule=0.4pt,left=5mm,right=2mm,top=1mm,bottom=1mm,
  colback=yellow!50,
  colframe=yellow!20!black,
  sharp corners,rounded corners=southeast,arc is angular,arc=3mm,
  underlay={%
    \path[fill=tcbcolback!80!black] ([yshift=3mm]interior.south east)--++(-0.4,-0.1)--++(0.1,-0.2);
    \path[draw=tcbcolframe,shorten <=-0.05mm,shorten >=-0.05mm] ([yshift=3mm]interior.south east)--++(-0.4,-0.1)--++(0.1,-0.2);
    \path[fill=yellow!50!black,draw=none] (interior.south west) rectangle node[white]{\Huge\bfseries !} ([xshift=4mm]interior.north west);
    },
  drop fuzzy shadow,#1}
\makeatother
%%%%%%%%%%%%%mybox0%%%%%%%%%%%%%%%
\newtcolorbox{mybox0}[2][]{enhanced,skin=enhancedlast jigsaw,attach boxed title to top left ={xshift=0mm, yshift=-3mm},fonttitle=\bfseries\sffamily,varwidth  boxed title=0.5\linewidth, colframe=teal,
	interior style={top color=teal!25!white,bottom color=violet!15!white},boxed title style={empty,arc=1pt,outer arc=1pt,boxrule=1pt},underlay boxed title={
		\fill[magenta!80!black] (title.north west) -- (title.north east)--(title.south east) -- (title.south west) -- cycle;
		\fill[red!40!orange] ([yshift=0.5mm]frame.north east)-- +(0,0) -- +(0,0.32) --+(-14.89,0.32)--+(-14.89,0) --cycle; },title={#2},#1}
	%%%%%%%%%%%%%%%%%
	\newtcolorbox{mybox}[2][]{enhanced,
	colback=blue!1,
	attach boxed title to top left={yshift*=1mm},
	varwidth boxed title*=-3cm,boxrule=1pt,breakable,colframe=red!50,drop lifted shadow,colback=blue!3}	
\tcolorboxenvironment{exercise}{
 	enhanced,
 	breakable,
 	arc=0pt,
 	outer arc=0pt,
 	rightrule=2pt,
 	toprule=2pt,
 	bottomrule=2pt,
 	leftrule=2pt,
 	colframe=red,
 	colback=pink!15 ,
 	attach boxed title to top left,drop lifted shadow,
 	boxed title style={colback=blue,arc=0pt,outer arc=0pt,colupper=white}
 }
 \newtcolorbox{Box2}[2][]{
                lower separated=false,
                colback=red!10,lifted shadow={1mm}{-2mm}{3mm}{0.1mm}%
{black!50!white},
colframe=white,fonttitle=\bfseries,
colbacktitle=red,
coltitle=white,
enhanced,
attach boxed title to top left={xshift=0em,yshift=-\tcboxedtitleheight/2,}, top=1.5ex,arc=0pt
boxed title style={boxrule=0pt,colframe=white,lifted shadow={1mm}{-2mm}{3mm}{0.1mm}%
{black!50},},
title=#2,#1}
 %%------------------newtheorem-------------------------
\newtheorem{exercise}{\kml លំហាត់ទី}
%%%--------------------
\renewcommand{\theexercise}{\kml \knum{exercise}}
%\renewcommand{\theex}{\knum{ex}}
\newtcbtheorem{ex}{ទ្រឹស្តីបទ}{
  enhanced,
  sharp corners,
  attach boxed title to top left={
    yshifttext=-1mm
  },
  colback=white,
  colframe=blue!75!black,
  fonttitle=\bfseries,
  boxed title style={
    sharp corners,
    size=small,
    colback=blue!75!black,
    colframe=blue!75!black,
  } 
}{thm}
\begin{document}
\renewcommand{\partname}{ផ្នែកទី} 
\renewcommand{\contentsname}{\color{cyan}{\kml មាតិកា}} 
%\tableofcontents
\def\chaptername{\kml ជំពូក}
	\begin{tikzpicture}[overlay,remember picture]
	\node at (10,0) {\includegraphics[width=37cm]{Bart-Simpson-Wallpaper1}};
	\draw [blue,fill=white!20,fill opacity = 0.5,thick,rounded corners=5ex] (30,-1) rectangle (5,1);
	\draw [blue,thick,rounded corners=5ex] (30,-1.2) rectangle (5,1.2);
	\draw [blue,thick,rounded corners=5ex] (30,-1.3) rectangle (5,1.3);
	\node[blue] at (10.05,0) {\fontsize{40}{100}\selectfont\kml ផ្នែកលំហាត់};
	\node[blue] at (10,0) {\fontsize{40}{100}\selectfont\kml ផ្នែកលំហាត់};
    \end{tikzpicture}
	\vspace*{2cm}
\begin{enumerate}[1]
\item  ស្រាយបញ្ជា់ថាផលបូកគូបនៃបីចំនួនគត់តគ្នាចែកដាច់នឹង​​$9$។
\item ស្រាយថាចំពោះគ្រប់$n$ជាចំនួនគត់
\begin{enumerate}[a]
\item $3n^4-14n^3+21n^2-10n \vdots 24$
\item $n^5-5n^3+4n \vdots 120$​ ។
\end{enumerate}
\item ចំពោះ$n$គូស្រាយបញ្ជាក់ថា$20^n+16^n-3n-1$ចែកដាច់នឹង$323$។
\item ស្រាយបញ្ជាក់ថាចំពោះគ្រប់$n$ជាចំនួនគត់ធម្មជាតិៈ
\begin{enumerate}[a]
\item $11^{n+2}+12^{2n+1} \vdots 133$
\item $5^{n+2}+26\times 5^n+8^{2n+1} \vdots 59$
\item $7\times5^{2n}+12\times6^n \vdots 19$។
\end{enumerate}
\item ស្រាយបញ្ជាក់ថា
\begin{enumerate}[a]
\item ${2222}^{5555}+{5555}^{2222} \vdots 7$
\item $4^{2n}-3^{2n}-7 \vdots 168,\forall n \in {\N}$
\item $2^{2^{2n}}+5 \vdots 7 ,\forall n \in {\N},n>1$
\end{enumerate}
\item គេឱ្យ $f(x)$ ជាពហុធាមានមេគុណជាចំនួ​នគត់ $f(x)=a_{n}x^{n}+a_{n-1}x^{n-1}+\cdots+a_{1}x+a_{o}$ ដែល ​$a_{i}\in{\Z}$\\ $i=0,1,2,\cdots,n$និង$a,b$ ជាពីរចំនួនគត់ផ្សេងគ្នា។​ ស្រាយថា$f(a)-f(b) \vdots a-b$\\ អនុវត្តៈ គ្មានពហុធា$P(x)$ណាមានមេគុណជាចំនួនគត់​អាចមានតម្លៃ$P(7)=5$និង$P(15)=9$។
\item ស្រាយបញ្ជាក់ថាបើ$n$មិនចែកដាច់$3$នោះ$3^{2n}+3^n+1 \vdots 13$ ។
\item រកគ្រប់ចំនួនគត់ធម្មជាតិ$n$ដើម្បីឱ្យ$2^n-1 \vdots 7$។ ស្រាយថាចំពោះ$\forall n \in{\N}$នោះ$2^n+1\not \vdots$  $7$។
\item ដោយប្រើវិចារអនុមានរួម​ស្រាយបញ្ជាក់ថាចំពោះគ្រប់$n$
\begin{enumerate}[a]
\item $16^n-15n-1 \vdots 225$
\item $3^{3n+3}-26n-27 \vdots 169$។
\end{enumerate}
\item ដោយប្រើវិចារអនុមានរួមស្រាយថាគ្រប់$n\in{\N}$
\begin{enumerate}[a]
\item $2^{2^{6n+2}}+3\vdots 19$
\item $2^{2^{2n+1}}+3 \vdots 7$  ។
\end{enumerate}
\item ដោយប្រើវិចារអនុមានរួមស្រាយបញ្ជាក់ថាចំពោះ$n\in{\N}$និង$k\in{\N}$,kសេសេ  គេបាន$k^{2^{n}}-1\vdots 2^{n+2}$។
\item ដោយប្រើវិចារអនុមានរួមស្រាយបញ្ជាក់ថា$\forall n \in n$
\begin{enumerate}[a]
\item $11^{n+2}+12^{2n+1}\vdots 133$
\item $6^{2n+1}+5^{n+2}\vdots 31$
\end{enumerate}
\item ស្រាយបញ្ជាក់ថា$x+\frac{1}{2}\in{\Z}$នោះ$x^n+\frac{1}{x^n}\in{\Z}$ចំពោះ$x\in{\N}$។. 
\item ស្រាយបញ្ជាក់ថា$1^{2002}+2^{2002}+3^{2002}+\cdots+2002^{2002} \vdots 11$
\item ស្រាយបញ្ជាក់ថា$3^{2^{4n+1}}+2^{3^{4n+1}}+5 \vdots 22$ចំពោះ$\forall n \in {\N}$។
\item ស្រាយបញ្ជាក់ថា$\forall n \in{\N}$
\begin{enumerate}[a]
\item $2^{2^{6n+2}}+3 \vdots 19$
\item $2^{2^{2n+1}}+3 \vdots 7$
\item $2^{2^{4n+1}}+7 \vdots 11$ ។
\end{enumerate}
\item ស្រាយញ្ជាក់ថា$a_1+a_2+...+a_{2018} \vdots 30$នោះ
${a_1}^5+{a_2}^5+...+a_{2018}^5 \vdots 30$ចំពោះ${a_i}\in{\N} ,i=1,2,...,2018$។
\item ស្រាយបញ្ជាក់ថា$2^{70}+3^{70} \vdots 13$។
\item $p$ជាចំនួនបឋមធំជាង$7$។ ស្រាយថា$3^p-2^p-2 \vdots 42p$។
\item រកសំណល់នៃវិធីចែក$2008^{2018}$នឹង$13$។
\item រកសំណល់នៃវិធីចែក${3^{{2^{2018}}}}$នឹង$11$ដោយ$gcd(3,11)=1$។
\item រកសំណល់នៃវិធីចែក$4362^{4362}$ ចែកនឹង$11$។
\item រកចំនួនគត់មានលេខពីរខ្ទង់ដែលពេលចែកនឹង$5$មានសំណល់$2$ចែកនឹង$9$មានសំណល់$7$។
\item រកសំណល់នៃវិធីចែក$2^{2017}$នឹង$35$។
\item រកសំណល់នៃវិធីចែក$3^{2^{4n+1}}$ចែកនឹង$65$។
\item រកសំណល់នៃវិធីចែក
\begin{enumerate}[a]
\item $3^{100}$នឹង$25$
\item $2^{2007}$នឹង$49$។
\end{enumerate}
\item រកសំណល់នៃវិធីចែក
\begin{enumerate}[a]
\item $3^{4^{15}}$នឹង$121$
\item $15^{15^{15}}$នឹង$49$។
\end{enumerate}
\item រកសំណលក្នុងវិធីចែក
\begin{enumerate}[a]
\begin{multicols}{2}
\item $6^{1991}$នឹង$28$
\item $35^{150}$នឹង$425$
\item $35^{12^{11}}$នឹង$325$
\item $22^{2002}$នឹង$1001$។
\end{multicols} 
\end{enumerate}
\item រកសំណល់នៃ$3^{2019}$ចែកនឹង$35$។
\item រកសំណល់នៃ$3^{2018}$ចែកនឹង$49$។
\item ដោយប្រើទ្រឹស្តីបទ$Euler$រកសំណល់នៃវិធីចែក$3^{2019}$នឹង$28$។
\item ដោយប្រើទ្រឹស្តីបទ{\en Euler}រកសំណល់ក្នុងវិធីចែក
\begin{enumerate}[a]
\item $109^{345}$នឹង$14$
\item $5^{70}+7^{50}$ចែកនឹង$12$។
\end{enumerate}
\item ដោយប្រើទ្រឹស្តីបទ{\en Euler}រកសំណល់នៃវិធីចែក
\begin{enumerate}[a]
\item $2008^{2008}$នឹង$35$
\item $11^{11^{11}}$នឹង$30$។
\end{enumerate}
%\newpage
\item ដោយប្រើ{\en Euler}រកសំណល់នៃវិធីចែក
\begin{enumerate}[a]
\item $3\times 5^{70}+4\times 7^{100}$នឹង$132$
\item $5^{70}+7^{50}$នឹង$12$។
\end{enumerate}
\item ស្រាយថា$0.3(2003^{2017}-2007^{2019})$ជាចំនួនគត់។
\item ស្រាយបញ្ជាក់ថា$2^{3^{4n+1}}+3 \vdots 11, \forall n \in{\N}$។
\item ស្រាយបញ្ជាក់ថា$9^{9^{9^{9^{9}}}}-9^{9^{9^{9}}} \vdots 10$។
\item រកលេខមួយខ្ទង់ចុងក្រោយនៃ$3^{2^{1998}}-2^{9^{1998}}$។
\item រកលេខមួយខ្ទងចុងក្រោយនៃផលចែកក្នុងវិធីចែក$3^{2^{1930}}+2^{9^{1945}}-19^{5^{1980}}$នឹង$7$។
\item រកលេខមួយខ្ទង់ចុងក្រោយនៃចំនួន$A=2017^{2018^{2019}}$។
\item រកលេខពីរខ្ទង់ចុងក្រោយនៃ$14^{14^{14}}$។
\item រកលេខពីរខ្ទង់ចុងក្រោយនៃ$2^{9^{1997}}$។
\item ស្រាយបញ្ជាក់ថា$9^{9^{9^{9}}}-9^{9^{9}}\vdots 100$។
\item រកលេខពីរខ្ទង់ចុងក្រោយនៃ$3^{2^{2008}}$។
\item ស្រាយបញ្ជាក់ថា$7^{2^{4n+1}}+4^{3^{4n+1}}-65$ចែកដាច់នឹង$100$។
\item រកលេខបីខ្ទង់ចុងក្រោយនៃ
\begin{enumerate}[a]
\begin{multicols}{2}
\item $2^{2008}$
\item $3^{2018}$។
\end{multicols}
\end{enumerate}
\item រកលេខបីខ្ទង់ចុងក្រោយនៃ
\begin{enumerate}[a]
\begin{multicols}{2}
\item $2^{9^{2001}}$
\item $14^{14^{14}}$
\item $3^{2^{2007}}$
\item $17^{5^{123}}$។
\end{multicols}
\end{enumerate}
\item រកពីរចំនួនគត់វិជ្ជមាន$a$និង$b$
\begin{multicols}{2}
\begin{enumerate}[a]
\item ដោយស្គាល់$\begin{cases}
	a+b=72\\gcd(a,b)=12
\end{cases}$
\item ដោយស្គាល់ $\begin{cases}
ab=735\\lcm(a,b)=105
\end{cases}$
\end{enumerate}
\end{multicols}
\item រក$n\in{\N}, n<100$ដែល$gcd(n,252)=7$។
\item ដោយដឹងថា$gcd(a,b)=1$រក$gcd(11a+2b,18a+5b)$។
\item រកចំនួនគត់ធម្មជាតិ$a$និង$b$ដោយដឹងថា
\begin{enumerate}[a]
\item $a+b=432,gcd(a,b)=36$
\item $a\times b=216,gcd(a,b)=6$
\item $ab=288,gcd(a,b)=6$។
\end{enumerate}
\item គេឱ្យប្រភាគសម្រួលមិនបាន$\dfrac{a}{b}$។ ស្រាយថាប្រភាគខាងក្រោមក៏សម្រួលមិនបានដែរ
\begin{multicols}{3}
\begin{enumerate}[a]
\item $\dfrac{a-b}{ab}$
\item $\dfrac{ab}{a^{2}+b^{2}}$
\item $\dfrac{2a+b}{a(a+b)}$ ។
\end{enumerate}
\end{multicols}
\item ស្រាយបញ្ជាក់ថាប្រភាគខាងក្រោមសម្រួលមិនបានចំពោះគ្រប់ចំនួនគត់$n$
\begin{multicols}{3}
\begin{enumerate}[a]
\item $\dfrac{12n+1}{30n+2}$
\item $\dfrac{15n^{2}+8n+6}{30n^{2}+21n+13}$
\item $\dfrac{n^{3}+2n}{n^{4}+3n^{2}+1}$ ។
\end{enumerate}
\end{multicols}
\item ស្រាយបញ្ជាក់ថាចំពោះ$a>1$និង$m>1$គេបាន$gcd\left(\dfrac{a^{m}-1}{a-1},a-1\right)=gcd(m,a-1)$ដែល $a,m\in{\N}$។
\item រកចំនួនគត់ធម្មជាតិ$n$តូចបំផុតដើម្បីឱ្យប្រភាគខាងក្រោមសម្រួលមិនបាន$\dfrac{7}{n+9}, \dfrac{8}{n+10}, \dfrac{9}{n+11},\cdots,\dfrac{31}{n+33}$​ ។
\item រកចំនួនគត់ធម្មជាត$n$ដើម្បី​ឱ្យ$\dfrac{6n+5}{5n+6}$សម្រួលមិនបាន។
\item រកគ្រប់ចំនួនគត់$n\in {\N}$ដើម្បីឱ្យ
\begin{enumerate}[a]
\item $n^{4}+4$ជាចំនួនបឋម
\item $n^{1997}+n^{1996}+1$ជាចំនួនបឋម
\item $n^{4}+n^{2}+1$ជាចំនួនបឋម។
\end{enumerate}
\item រកចំនួនបឋម$p$ដែល
\begin{enumerate}[a]
\item $2p+1$ជាគូបនៃចំនួគត់ធម្មជាតិ
\item $13p+1$ជាគូបនៃចំនួនគត់ធម្មជាតិ។
\end{enumerate}
\item ស្រាយបញ្ជាក់ថា$2^{2{10n+1}}+19$និង$2^{3^{4n+1}}+3^{2^{4n+1}}+5$ជាចំនួនសមាសចំពោះ$\forall n \in {\N}$។
\item រកគ្រប់ចំនួនគត់វិជ្ជមាន$a$និង$b$ដើម្បីឱ្យ$a^{4}+4b^{4}$ជាចំនួនបឋម។
\item $p$ជាចំនួនបឋមធំជាង$5$ស្រាយបញ្ជាក់ថា$p^{8n}+3p^{4n}-4 \vdots 5$។
\item រកចំនួនបឋម$p$ដើម្បីឱ្យ$2^{p}+p^{2}$ក៏ជាចំនួនបឋមដែរ។
\item រកគ្រប់ឬសគត់វិជ្ជមាន $x,y$ ដែលផ្ទៀងផ្ទាត់សមីការៈ
\begin{enumerate}[a]
\begin{multicols}{2}
\item $5x+7y=112$
\item $5x+19y=674$
\item $38x+117y=15$
\item $21x-17y=-3$
\end{multicols}
\end{enumerate}
\item 
\begin{enumerate}[a]
\item នៅលើបន្ទាត់​ $8x-13y+6=0$ ចូររកចំណុចគត់(ជាចំណុចមានកូអរដោនេជាចំនួនគត់)នៅចន្លោះបន្ទាត់ពីរ\\ $x=-10$​ និង $x=5$។
\item ស្រាយបញ្ជាក់ថាក្នុងចតុកោណកែងកំណត់ដោយបន្ទាត់ $x=6, x=42$ និង $y=2, y=17$ មិនមានចំណុចគត់ណានៅលើបន្ទាត់ $3x+5y=7$។
\end{enumerate}
\item 
\begin{enumerate}[a]
\item រកឬសគត់នៃសមីការ $x+y=xy$ ។
\item​ រកពីរចំនួនគត់ធម្មជាតិដែលផលដកការេរបស់វាស្មើនឹង $169$។
\end{enumerate}
\item
\begin{enumerate}[a]
\item រកគ្រប់បណ្តាត្រីកោណកែងដែលមានជ្រុងជាចំនួនគត់ហើយមានផ្ទៃក្រឡាស្មើបរិមាត្រ។
\item រកឬសជាចំនួនគត់ធម្មជាតិនៃ $3x^3-xy=5$។
\end{enumerate}
\item រកគ្រប់ឬសគត់វិជ្ជមាននៃសមីការ $x+y+z=xyz$។
\item ត្រីកោណមួយមានកម្ពស់ជាចំនួនគត់និងកាំរង្វង់ចារឹកក្នុងត្រីកោណនេះស្មើ $1$​ ។ ស្រាយបញ្ជាក់ថាត្រីកោណនេះជាត្រីកោណសម័ង្ស។
\item រកគ្រប់ចំនួនគត់មានលេខបីខ្ទង់ដែលផលបូកចម្រាស់នៃខ្ទង់របស់វាស្មើនឹង​$1$។
\item រកឬសគត់វិជ្ជមាននៃប្រព័ន្ធសមីការខាងក្រោម៖
\begin{enumerate}[a]
\begin{multicols}{2}
\item $\begin{cases}x-y+z=2\\2x^2-xy+x-2z=1\end{cases}$
\item $\begin{cases}
	3x+y+z=14\\5x+3y+z=28
\end{cases}$
\end{multicols}
\end{enumerate}
\item រកឬសគត់នៃប្រព័ន្ធសមីការខាងក្រោម៖
\begin{multicols}{2}
\begin{enumerate}[a]
\item $\begin{cases}x^2-y^2-z^2=1\\-x+y+z=3\end{cases}$
\item $\begin{cases}x+y+z=3\\(z+y)(y-3)(z-3)=8\end{cases}$
\end{enumerate}
\end{multicols}
\item 
\begin{enumerate}[a]
\item រកឬសគត់វិជ្ជមាននៃសមីការ $\frac{xy}{z}+\frac{yz}{x}+\frac{xz}{y}=3$
\item រកឬសគត់វិជ្ជមាននៃសមីការ​ $\left(x+y+1\right)^2=3\left(x^2+y^2+1\right)$​។
\end{enumerate}
\item 
\begin{enumerate}[a]
\item រកឬសគត់នៃសមីការ $x^2-2y^2=5$
\item ដោះស្រាយសមីការក្នុងសំណុំចំនួនគត់ $x_1^4+x_2^4+\cdots+x_7^4=2008$
\end{enumerate}
\item រកឬសគត់វិជ្ជមាននៃសមីការ $\left(2x+5y+1\right)\left(2^{\vert x \vert}+y+x^2+x\right)=105$។
\item រកឬសគត់វិជ្ជមាននៃសមីការ $2\left(x+y\right)+16=3xy$​។
\item រកចំនួនជាការេប្រាកដដែលមានរាង​ $\overline{22ab}$ ។
\item រក $a\in {\N}$ ដើម្បីឱ្យចំនួនខាងក្រោមជាការេប្រាកដៈ
\begin{enumerate}[a]
\item $a^2+a+1589$
\item $a^2+81$ ។
\end{enumerate}
\item 
\begin{enumerate}[a]
\item រកចំនួនគត់ $n\in {\N}$ ដែល $n+24$ និង​ $n-65$ ជាពីរចំនួនការេប្រាកដ។
\item​ រក $n\in {\Z}$ ដែល $n^2+2007$​ ជាការេប្រាកដ។
\end{enumerate}
\item រកចំនួនមួយជាការេប្រាកដមានលេខបួនខ្ទង់ដែលខ្ទង់ចុងក្រោយជាចំនួនបឋម​ ហើយឬសការេនៃចំណុចនោះមានផលបូកជាការេប្រាកដ។
\item រកចំនួនគត់ធម្មជាតិដែលៈ
\begin{enumerate}[a]
\item បើថែម $64$ឬបន្ថយ$35$គេសុទ្ធតែបានចំនួនជាការេប្រាកដ។
\item បើគេថែម$51$ឬបន្ថយ$38$គេនៅតែបានចំនួនជាការេប្រាកដ។
\end{enumerate}
\item រកចំនួនជាការេប្រាកដមានលេខបួនខ្ទង់ដោយដឹងថាពីរខ្ទង់ដំបូងធំជាងលេខខ្ទង់ចុងក្រោយចំនួនមួយឯកតា។
\item​ រកមួយចំនួនជាការេប្រាកដមានលេខបីខ្ទង់ចុងក្រោយហើយចែកដាច់នឹង$56$។
\item គណនាតម្លៃនៃកន្សោម $A=\log_n \sqrt{n \sqrt[3]{n^2\sqrt[4]{n^3\cdots \sqrt[n]{n^{n-1}}}}}$ ។
\item កំណត់តម្លៃ $x$ ប្រសិនបើ $3^x-2^x=\left(3+2\right)\left(3^2+2^2\right)\left(3^4+2^4\right)\cdots \left(3^{32}+2^{32}\right)$ ។
\item គណនាតម្លៃនៃ $5\sqrt{3\sqrt{5\sqrt{3\cdots}}}$។
\item 
\begin{enumerate}[a]
\item រកសំណល់នៃវិធីចែកតាមបែបអឺគ្លីតនៃ $2^{2015}$ នឹង $7$។
\item រកចំនួនគត់វិជ្ជមាន $k$ ដែល $2^{2015}+k \vdots 7$។
\end{enumerate}
\item បង្ហាញថា $\log_2 \cos20^{\circ}+\log_2 \cos40^{\circ}+\log_2\cos80^{\circ}=-3$។
\item គណនាតម្លៃ $10\sqrt{10\sqrt{10\sqrt{10\cdots}}}$។
\item បង្ហាញថា $A=1^n+8^n-3^n-6^n$​ ចែកដាច់នឹង $10$ គ្រប់តម្លៃ$n$។
\item គេឱ្យ $f_0(x)=\frac{1}{x-1}$ ហើយ $f_n(x)=f_0\left(f_{n-1}(x)\right)$​ដែល $n=1,2,3,\cdots$ ។​ គណនាតម្លៃនៃ $f_{2019}(2019)$។​
\item បង្ហាញថា $A=2+2^2+2^3+\cdots+2^{60}$ ចែកដាច់នឹង $105$។
\item ស្រាយបញ្ជាក់ថា $2009^{3^{2016n+2013}}+2010^{2^{2016n+2013}}$ចែកដាច់នឹង $11$។
\item រកលេខបីខ្ទង់ចុងក្រោយនៃ $1993^{1994^{1995^{\cdots^{10000}}}}$។
\item បង្ហាញថាចំនួន $A=2^{2^{2016}}+2^{2^{2015}}+1$​ ចែកដាច់នឹង $21$។
\item រកសំណល់ពេល $\left(n^2+n+41\right)^2$ចែកនឹង $12$។
\item រកលេខខ្ទង់ចុងក្រោយនៃផលបូក $A=1+\underbrace{2^{2^{\cdots^{2}}}}_{2015\text{ដង}}+\underbrace{3^{3^{\cdots^{3}}}}_{2015\text{ដង}}+\underbrace{4^{4^{\cdots^{4}}}}_{2015\text{ដង}}+\underbrace{5^{5^{\cdots^{5}}}}_{2015\text{ដង}}+\underbrace{6^{6^{\cdots^{6}}}}_{2015\text{ដង}}+\underbrace{7^{7^{\cdots^{7}}}}_{2015\text{ដង}}$ ។
\item ស្រាយបញ្ជាក់សមភាព
\begin{enumerate}[a]
\item $kC_n^k=nC_{n-1}^{k-1}$
\item $nC_{2n}^{n}=(n+1)C_{2n}^{n+1}$
\end{enumerate}
\item រកតួទី$13$នៃទ្វេធាញូតុន $\left(\sqrt[3]{3}-\sqrt{15}\right)^{15}$។
\item ក្នុងទ្វេធាញូតុន $\left(3x^3-\frac{2}{x^2}\right)^{2015}$  រកមេគុណនៃតួមាន $x^{10}$។
\item សម្រួលកន្សោម $H=C_n^0+2C_n^1+3C_n^2+\cdots+(n+1)C_n^n$
\item គណនាផលបូក
\begin{enumerate}[A]
\item $=C_n^0+3C_n^1+3^2C_n^2+\cdots+3^nC_n^n$
\item $=C_{2n}^2+C_{2n}^4+\cdots+C_{2n}^{2n}$
\item $=2C_n^1+2^2C_n^2+\cdots+2^nC_n^n$
\item $=1-2C_n^1+2^2C_n^2-2^3C_n^3+\cdots+(-1)^n2^nC_n^n$ ។
\end{enumerate}
\end{enumerate}
\newpage
\begin{tikzpicture}[overlay,remember picture]
\node at (10,0) {\includegraphics[width=38cm]{Bart-Simpson-Wallpaper2}};
	\draw [black!70,fill=red!10,fill opacity = 0.5,thick,rounded corners=5ex] (30,-1) rectangle (5,1);
	\draw [black!70,thick,rounded corners=5ex] (30,-1.2) rectangle (5,1.2);
	\draw [black!70,thick,rounded corners=5ex] (30,-1.3) rectangle (5,1.3);
	\node[white] at (11,0) {\fontsize{40}{100}\selectfont\kml ផ្នែកដំណោះស្រាយ};
	\node[black!80] at (11.07,0) {\fontsize{40}{100}\selectfont\kml ផ្នែកដំណោះស្រាយ};
\end{tikzpicture}
	\vspace*{2.6cm}
\begin{exercise}
ស្រាយបញ្ជាក់ថាផលបូកគូបនៃបីចំនួនគត់តគ្នាចែកដាច់នឹង​​$9$។
\end{exercise}
\begin{proof}[\solution]
ស្រាយបញ្ជាក់ថាផលបូកគូបនៃបីចំនួនគត់តគ្នាចែកដាច់នឹង​​$9$\\
តាង $n-1, n,n+1$​ជាបីចំនួនគត់តគ្នា\\
តាង $A$​ជាផលបូកនៃបីចំនួនគត់តគ្នា\\
$\begin{aligned}
\text{គេបាន}​  A&=\left(n-1\right)^3+n^3+\left(n+1\right)^3=n^3-3n^2+3n-1+n^3+3n^2+3n+1\\
&=3n^3+6n=3n\left(n^2+2\right)=3n\left(n^2-1+3\right)\\
&=3n\left(n^2-1\right)+9n=3n\left(n-1\right)\left(n+1\right)+9n
\end{aligned}$\\
ដោយ$9n\vdots 9$\\
ហើយ $n(n-1)(n+1)\vdots3!\vdots6$
 \;
 \begin{tikzpicture}
 \node [rectangle callout,draw=magenta!70,fill=magenta!70,callout absolute pointer={(0,1)}
] at (5,1) {\kml រូបមន្ត​ $n(n+1)(n+2)\cdots(n+k)\vdots k!$};
 \end{tikzpicture}\\
នោះ $3n(n-1)(n+1)\vdots 9$\\
នោះ $A\vdots9$ (ពិត)\\
\fbox{\so  ផលបូកគូបនៃបីចំនួនគត់តគ្នាចែកដាច់នឹង​​$9$}
\end{proof}
\begin{exercise}
ស្រាយថាចំពោះគ្រប់$n$ជាចំនួនគត់
\begin{enumerate}[a]
\item $3n^4-14n^3+21n^2-10n \vdots 24$
\item $n^5-5n^3+4n \vdots 120$​ ។
\end{enumerate}
\end{exercise}
\begin{proof}[\solution]
ស្រាយថាចំពោះគ្រប់$n$ជាចំនួនគត់
\begin{enumerate}[a]
\item  $3n^4-14n^3+21n^2-10n \vdots 24$\\
តាង $B= 3n^4-14n^3+21n^2-10n$\\
$\begin{aligned}[t]
\text{គេបាន}\;​ B&=n\left(3n^3-14n^2+21n-10\right)=n\left(3n^3-3n^2-11n^2+11n+10n-10\right)\\
&=n\left[3n^2\left(n-1\right)-11n(n-1)+10(n-1)\right]=n(n-1)\left(3n^2-11n+10\right)\\
&=n(n-1)\left(3n^2-6n-5n+10\right)=n(n-1)\left[3n(n-2)-5(n-1)\right]\\
&=n(n-1)(n-2)(3n-5)=n(n-1)(n-2)(3n-9+4)\end{aligned}$\\
$B=3n(n-1)(n-2)(n-3)+4n(n-1)(n-2)
$\\
ដោយ $n(n-1)(n-2)(n-3)=4!\vdots24$\\
ហើយ $n(n-1)(n-2)\vdots3!\vdots6$\\
$\Rightarrow 3n(n-1)(n-2)=3\times 6=24\vdots 24$\\
$\Rightarrow B\vdots 24$\qquad (ពិត)\\
\fbox{\so  $3n^4-14n^3+21n^2-10n \vdots 24$}
\item $n^5-5n^3+4n \vdots 120$\\
តាង $C=n^5-5n^3+4n$\\
 $\begin{aligned}[t] 
 \text{គេបាន} C&=n\left(n^4-5n^2+4\right)=n\left(n^4-2n^3+2n^3-4n^2-n^2+2n-2n+4\right)\\
 &=n\left[n^3(n-2)+2n^2(n-2)-n(n-2)-2(n-2)\right]=n(n-2)\left(n^3+2n^2-n-2\right)\\
 &=n(n-2)\left[n^2(n+2)-(n+2)\right]=n(n-2)(n+2)\left(n^2-1\right)\\
 &=n(n-2)(n+2)(n-1)(n+1)=n(n-1)(n-2)(n+1)(n+2)\\
  \end{aligned}$\\
ដោយ $ n(n-1)(n-2)(n+1)(n+2)\vdots 5!\vdots 120$\\
នោះ $B\vdots 120$\quad (ពិត)\\
\fbox{\so $n^5-5n^3+4n$ចែកដាច់នឹង $120$}
\end{enumerate}
\end{proof}
\begin{exercise}
 ចំពោះ$n$គូស្រាយបញ្ជាក់ថា$20^n+16^n-3n-1$ចែកដាច់នឹង$323$។
\end{exercise}
\begin{proof}[\solution]
ស្រាយបញ្ជាក់ថា$20^n+16^n-3n-1$ចែកដាច់នឹង$323$\\
តាង $S_n=20^n+16^n-3^n-1$\\
ដោយ​ $323=17\times19$\\
គេបាន$\begin{cases}20^n-3^n\vdots20-3\quad ,\forall n,n\text{គត់}\\16^n-1\vdots16+1\quad,\forall n,n\text{គូ}\end{cases}$
 $\Rightarrow S_n\vdots 17$\quad (1)\\
ម្យ៉ាងទៀត\\
$\begin{cases}20^n-1\vdots20-1\quad,\forall n,n\text{គត់}\\16^n-3^n\vdots 16+3\quad,\forall n,n\text{គូ}\end{cases}$
$\Rightarrow S_n\vdots19\quad(2)$\\
តាម (1)និង (2)គេបាន\\
$S_n\vdots323$\quad (ពិត)\\
\fbox{\so$20^n+16^n-3n-1$ចែកដាច់នឹង$323$} 
\end{proof}
\begin{exercise}
ស្រាយបញ្ជាក់ថាចំពោះគ្រប់$n$ជាចំនួនគត់ធម្មជាតិៈ
\begin{enumerate}[a]
\item $11^{n+2}+12^{2n+1} \vdots 133$
\item $5^{n+2}+26\times 5^n+8^{2n+1} \vdots 59$
\item $7\times5^{2n}+12\times6^n \vdots 19$។
\end{enumerate}
\end{exercise}
\begin{proof}[\solution]
ស្រាយបញ្ជាក់ថា
\begin{enumerate}[a]
\item $11^{n+2}+12^{2n+1} \vdots 133$\\
តាង $B_n=11^{n+2}+12^{2n+1} $\\
$\begin{aligned}
\text{គេបាន} B_n&=11^2\cdot11^n+\left(12^2\right)^n\cdot12\\
&=121\cdot11^n+144^n\cdot12=121\cdot11^n+\left(133+11\right)^n\cdot12\\
&\equiv 121\cdot11^n+11^n\cdot12\equiv 133\cdot11^n\quad (mod 133)\end{aligned}$\\
$
B_n\equiv 0\quad (mod 133)
$\\
$\Rightarrow B_n\vdots133$\quad (ពិត)\\
\fbox{\so $11^{n+2}+12^{2n+1} \vdots 133$}
\item $5^{n+2}+26\times 5^n+8^{2n+1} \vdots 59$\\
តាង $C=5^{n+2}+26\times 5^n+8^{2n+1}$\\
$\begin{aligned}[t]
\text{គេបាន} C&=25\cdot5^n+26\cdot5^n+8\cdot8^{2n}\\
&=25\cdot5^n+26\cdot5^n+64^n\cdot8=51\cdot5^n+(59+5)^n\cdot8\\
&\equiv 51\cdot5^n+5^n\cdot8\equiv 59\cdot5^n\equiv 0\quad(mod59)\\
\end{aligned}$\\
$\Rightarrow C\vdots59$\quad(ពិត)\\
\fbox{\so $5^{n+2}+26\times 5^n+8^{2n+1} \vdots 59$}
\item $7\times5^{2n}+12\times6^n \vdots 19$\\
គេបាន$\begin{aligned}[t]
&7\times5^{2n}+12\times6^n=7\times25^n+12\times6^n
=7\times(19+6)^n+12\times6^n\\
&\equiv 7\times6^n+12\times6^n\equiv 19\times6^n\quad(mod19)
\end{aligned}$\\
$\Rightarrow 7\times5^{2n}+12\times6^n \vdots 19$\quad(ពិត)\\
\fbox{\so $7\times5^{2n}+12\times6^n \vdots 19$}
\end{enumerate}
\end{proof}
\begin{exercise}
ស្រាយបញ្ជាក់ថា
\begin{enumerate}[a]
\item ${2222}^{5555}+{5555}^{2222} \vdots 7$
\item $4^{2n}-3^{2n}-7 \vdots 168,\forall n \in {\N}$
\item $2^{2^{2n}}+5 \vdots 7 ,\forall n \in {\N},n>1$
\end{enumerate}
\end{exercise}
\begin{proof}[\solution]
ស្រាយបញ្ជាក់ថា
\begin{enumerate}[a]
\item ${2222}^{5555}+{5555}^{2222} \vdots 7$\\
ដោយ $2222=7\times317+3$\\
ហើយ$5555=7\times794-3$\\
គេបាន$\begin{aligned}[t]
&{2222}^{5555}+{5555}^{2222}=\left(7\times317+3\right)^{5555}+\left(7\times794-3\right)^{2222}\\
&\equiv 3^{5555}+3^{2222}\equiv 3^{2222}\left(3^{3333}+1\right)\quad(mod7)\\
&\equiv 3^{2222}\left[\left(3^3\right)^{1111}+1\right]\equiv 3^{2222}\left(27^{1111}+1\right)\quad(mod7)\\
&\equiv 3^{2222}\left[(28-1)^{1111}+1\right]\equiv 3^{2222}\left[(-1)^{1111}+1\right]\quad(mod7)\\
&\equiv 3^{2222}(-1+1)\equiv 0\quad(mod7)
\end{aligned}$\\
$\Rightarrow{2222}^{5555}+{5555}^{2222} \vdots 7$\quad (ពិត)\\
\fbox{\so ${2222}^{5555}+{5555}^{2222} \vdots 7$}
\item $4^{2n}-3^{2n}-7 \vdots 168,\forall n \in {\N}$\\
តាង $S=4^{2n}-3^{2n}-7 \vdots 168$\\
ដោយ $168=3\times7\times8$\\
+បង្ហាញថា $S\vdots3$\\
$\begin{aligned}[t]\Rightarrow S&=4^{2n}-3^{2n}-7=16^n-9^n-9^n-7=(3\times5+1)^n-9^n-(6+1)\\
&\equiv 1^n-0-1\equiv 0\quad(mod3)\end{aligned}$\\
$\Rightarrow S\vdots3\quad(1)$\\
+បង្ហាញថា $S\vdots7$\\
$\begin{aligned}[t]
\Rightarrow S&=4^{2n}-3^{2n}-7=16^n-9^n-7=(2\times7+2)^n-(7+2)^n-7\\
&\equiv 2^n-2^n-0\equiv 0\quad(mod7)
\end{aligned}$\\
$\Rightarrow S\vdots 7\quad (2)$\\
+បង្ហាញថា $S\vdots 8$\\
$\begin{aligned}[t]
\Rightarrow S&=4^{2n}-3^{2n}-7=16^n-9^n-7=16^n-(8+1)^n-(8-1)\\
&\equiv -1+1\equiv 0\quad(mod8)
\end{aligned}$\\
$\Rightarrow S\vdots8\quad(3)$\\
តាម $(1),(2)$និង $(3)$គេបានៈ\\
$S\vdots168$\quad(ពិត)\\
\fbox{\so  $4^{2n}-3^{2n}-7 \vdots 168,\forall n \in {\N}$}
\item $2^{2^{2n}}+5 \vdots 7 ,\forall n \in {\N},n>1$\\
ដោយ $2^3\equiv1(mod7)$\\
យក $2^{2n}$ចែកនឹង $3$គេបានៈ\\
$2^{2n}=4^n=(3+1)^n\\
\equiv 1(mod3)\\
\Rightarrow 2^{2n}=3k+1\\
\Rightarrow2^{2^{2n}}+5=2^{3k+1}+5\\
=2\cdot8^k+5=2(7+1)^k+5\\
\equiv 2+7\equiv 0(mod7)$\\
$\Rightarrow 2^{2^{2n}}+5\vdots7$\quad(ពិត)\\
\fbox{\so $2^{2^{2n}}+5 \vdots 7 ,\forall n \in {\N},n>1$}
\end{enumerate}
\end{proof}
\begin{exercise}
គេឱ្យ $f(x)$ ជាពហុធាមានមេគុណជាចំនួ​នគត់ $f(x)=a_{n}x^{n}+a_{n-1}x^{n-1}+\cdots+a_{1}x+a_{o}$ ដែល ​$a_{i}\in{\Z}$\\ $i=0,1,2,\cdots,n$និង$a,b$ ជាពីរចំនួនគត់ផ្សេងគ្នា។​ ស្រាយថា$f(a)-f(b) \vdots a-b$\\ អនុវត្តៈ គ្មានពហុធា$P(x)$ណាមានមេគុណជាចំនួនគត់​អាចមានតម្លៃ$P(7)=5$និង$P(15)=9$។
\end{exercise}
\begin{proof}[\solution]
ស្រាយបញ្ជាក់ថា $f(a)-f(b)\vdots a-b$\\
យើងមាន $f(x)=a_{n}x^{n}+a_{n-1}x^{n-1}+\cdots+a_{1}x+a_{o}$\\
គេបានៈ \\
$
-\underline{\begin{cases}
f(x)=a_na^n+a_{n-1}a^{n-1}+\cdots+a_1a+a_0\\
f(b)=a_nb^n+a_{n-1}b^{n-1}+\cdots+a_1b+a_0
\end{cases}}$\\
$\Rightarrow f(a)-f(b)=a_n\left(a^n-b^n\right)+a_{n-1}\left(a^{n-1}-b^{n-1}\right)+\cdots+a_1\left(a-b\right)$\\
ដោយ $\begin{aligned}[t]&a^n-b^n\vdots a-b\; ,\forall k\in \Z\\
&a^{n-1}-b^{n-1}\vdots a-b\\
&a-b\vdots a-b 
\end{aligned}$\\
នោះ $f(a)-f(b)\vdots a-b$\quad(ពិត)\\
\fbox{\so $f(a)-f(b)\vdots​ a-b$}\\
$\looparrowright$ស្រាយថាគ្មានពហុធា$P(x)$ណាមានមេគុណជាចំនួនគត់អាចមានតម្លៃ$P(7)=5$និង $P(15)=9$\\
ឧបមាថា $P(x)$មានមេគុណជាចំនួនគត់\\
តាមសម្រាយបញ្ជាក់ខាងលើគេបាន\\
$\begin{aligned}[t]
P(15)-P(7)&\vdots 15-7\\
9-5&\vdots 8\\
4&\vdots 8\quad\text{(មិនពិត)}
\end{aligned}$\quad(មិនពិត)\\
\fbox{\so គ្មានពហុធា$P(x)$ណាមានមេគុណជាចំនួនគត់អាចមានតម្លៃ$P(7)=5$និង $P(15)=9$}
\end{proof}
\begin{exercise}
 ស្រាយបញ្ជាក់ថាបើ$n$មិនចែកដាច់$3$នោះ$3^{2n}+3^n+1 \vdots 13$ ។
\end{exercise}
\begin{proof}[\solution]
ស្រាយបញ្ជាក់ថា $3^{2n}+3^n+1 \vdots 13 $\\
ដោយ​ $n$ ចែកមិនដាច់នឹង $3$\\
តាង $n=3k+r\quad ,r=1,2$\\
តាង $P(n)=3^{2n}+3^n+1 \vdots 13$\\
$\begin{aligned}[t]
\text{គេបាន} P(n)&=3^{2(3k+r)}+3^{3k+r}+1=3^{2\times 3k}\cdot 3^{2r}+3^{3k}\cdot3^r+1\\
&=27^{2k}\cdot 3^{2r}+27^k\cdot3^r+1=(26+1)^{2k}\cdot 3^{2r}+(26+1)^k\cdot3^r+1\\
&\equiv 3^{2r}+3^r+1\quad(mod 13)\\
\end{aligned}$\\
$\looparrowright$បើ​$r=1$\\
$\begin{aligned}[t]
\Rightarrow P(n)&\equiv3^{2\times1}+3^1+1\equiv 9+3+1\quad(mod13)\\
&\equiv 13\equiv 0\quad(mod 13)
\end{aligned}$\\
$\looparrowright \text{បើ}r=2$\\
$\begin{aligned}[t]
\Rightarrow P(n)&\equiv 3^{2\times2}+3^2+1\equiv 81+9+1\quad(mod 13)\\
&\equiv 91\equiv 0\quad(mod 13)
\end{aligned}$\\
$\Rightarrow 3^{2n}+3^n+1 \vdots 13$\quad  (ពិត)\\
\fbox{\so $3^{2n}+3^n+1 \vdots 13$}
\end{proof}
\begin{exercise}
រកគ្រប់ចំនួនគត់ធម្មជាតិ$n$ដើម្បីឱ្យ$2^n-1 \vdots 7$។ ស្រាយថាចំពោះ$\forall n \in{\N}$\\នោះ$2^n+1\not \vdots$  $7$។
\end{exercise}
\begin{proof}[\solution]
រកគ្រប់ចំនួនគត់ធម្មជាតិ$n$ដើម្បីឱ្យ$2^n-1 \vdots 7$\\
 ដោយ $8=2^3\equiv 1\quad(mod 7)$\\
  $\begin{aligned}[t]
 \text{តាង} &n=3k\Rightarrow 2^{3k}-1\equiv 1-1\equiv 0\quad(mod 7)
 \end{aligned}$\\
  $\begin{aligned}[t]
 \text{តាង} n&=3k+1\Rightarrow 2^{3k+1}-1=2^{3k}\cdot2-1\\
  &\equiv 2-1\equiv 1\quad(mod 7)
  \end{aligned}$\\
  $\begin{aligned}[t]
  \text{តាង}&n=3k+2\Rightarrow 2^{3k+2}-1=2^{3k}\cdot 4-1\\
  &\equiv 4-1\equiv 3\quad(mod 7)
 \end{aligned}$\\
  \fbox{\so $n=3k$ ដែល $k\in{\N}$ធ្វើឱ្យ $2^n-1\vdots 7$}\\
  +ស្រាយថាចំពោះ$\forall n \in{\N}$នោះ$2^n+1\not \vdots$  $7$\\
តាង $n=3k+r ,r=0,1,2$\\
$\begin{aligned}[t]
2^n+1&=2^{3k+r}+1=2^{3k}\cdot 2^r+1\\
&\equiv 2^r+1\quad(mod 7)
\end{aligned}$\\
$\begin{aligned}[t]
\text{-បើ} r=0\Rightarrow 2^n+1&\equiv 2^0+1\\
&\equiv 2\quad(mod7)
\end{aligned}$\\
-បើ​$r=1\Rightarrow 2^n+1\equiv 2^1+1\equiv 3\quad(mod 7)$\\
-បើ $r=2\Rightarrow 2^n+1\equiv 2^2+1\equiv 5\quad(mod7)$\\
នោះ$2^n+1\not \vdots7$\quad(ពិត)\\
\fbox{\so $2^n+1\not \vdots7,\forall n \in {\N}$}
 \end{proof}
 \begin{exercise}
 ដោយប្រើវិចារអនុមានរួម​ស្រាយបញ្ជាក់ថាចំពោះគ្រប់$n$
\begin{enumerate}[a]
\item $16^n-15n-1 \vdots 225$
\item $3^{3n+3}-26n-27 \vdots 169$។
\end{enumerate}
\end{exercise}
\begin{proof}[\solution]
ស្រាយបញ្ជាក់ថាចំពោះ $\forall n\in{\N}$
\begin{enumerate}[a]
\item  $16^n-15n-1\vdots 225$\\
តាង​ $f(n)=16^n-15n-1$\\
$\begin{aligned}[t]
\text{បើ} n=1\Rightarrow f(1)&=16^1-15\times 1-1\\
&=0\vdots 225
\end{aligned} $\\
$\star$ឧបមាវាពិតដល់ $n=k\Rightarrow f(k)\vdots 225\quad,k>1$\\
$f(k)=16^k-15k-1\vdots 225\\
16^k-15k-1=225t\quad t\in{\N}\\
16^k=225t+15k+1$\\
$\star$ស្រាយថាវាពិតដល់​ $n=k+1\Rightarrow f(k+1)\vdots 225$\\
$\begin{aligned}[t]
f(k+1)&=16^{k+1}-15(k+1)-1\\
&=16\times 16^k-15k-15-1=16(225t+15k+1)-15k-16\\
&=16\times 225t+16\times15k+16-15k-16=16\times 225t+16\times 15k-15k\end{aligned}$\\
$f(k+1)= 16\times 225t+15\times 15k=16\times 225t+225k=225(16t+k)\vdots 225$\\
នោះ $f(n)\vdots 225$\quad(ពិត)\\
\fbox{\so $16^n-15n-1 \vdots 225$}
\item $3^{3n+3}-26n-27 \vdots 169$\\
តាង​ $P(n)=3^{3n+3}-26n-27$\\
$\begin{aligned}[t]
\text{បើ}n=1\Rightarrow P(1)&=3^6-26-27\\
&=676\vdots 169
\end{aligned}$\\
នោះ $P(1)\vdots 169$\quad(ពិត)\\
ឧបមាវាពិតដល់ $n=k\Rightarrow P(k)\vdots 169\quad ,k>1$\\
$\Rightarrow P(k)=3^{3k+3}-26\cdot k-27\vdots 169\\
3^{3k+3}-26k-27=169t\quad ,t\in{\N}\\
3^{3k+3}=169t+26k+27$\\
ស្រាយវាថាពិតដល់ $n=k+1\Rightarrow P(k+1)\vdots 169$គេបានៈ\\
$\begin{aligned}[t]
 P(k+1)&=3^{3k+3+3}-26(k+1)-27=27\left(3^{3k+3}\right)-26k-53\\
&=27(169t+26k+27)-26k-53=27\times 169t+27\times 26k+27\times 27-26k-53\\
&=27\times 169t+27\times 26k+729-53=27\times 169t+676k+676\\
&=27\times 169t+169\times 4k+169\times 4
\end{aligned}$\\
$P(k+1)=169(27t+4k+4)\vdots 169$\\
$\Rightarrow P(k+1)\vdots 169$\quad (ពិត)\\
\fbox{\so $3^{3n+3}-26n-27 \vdots 169$}
\end{enumerate}
\end{proof}
\begin{exercise}
ដោយប្រើវិចារអនុមានរួមស្រាយថាគ្រប់$n\in{\N}$
\begin{enumerate}[a]
\item $2^{2^{6n+2}}+3\vdots 19$
\item $2^{2^{2n+1}}+3 \vdots 7$  ។
\end{enumerate}
\end{exercise}
\begin{proof}[\solution]
ស្រាយថាគ្រប់$n\in{\N}$
\begin{enumerate}[a]
\item $2^{2^{6n+2}}+3\vdots 19$\\
តាង $f(n)=2^{2^{6n+2}}+3$\\
បើ $n=1$គេបាន\\
$\begin{aligned}[t]
f(1)&=2^{2^8}+3=2^{256}+3=\left(2^4\right)^{64}+3=(19-3)^{64}+3\\
&\equiv 3^{64}+3\equiv \left(3^4\right)^{64}+3\equiv (19\times 4+5)^{16}+3\quad(mod19)\\
&\equiv 5^{16}+3\equiv (25)^8+3\equiv (19+6)^8+3\quad(mod 19)\\
&\equiv 6^8+3\equiv (36)^4+3\equiv (38-2)^4\quad (mod 19)\\
&\equiv 2^4+3\equiv 19\equiv 0\quad(mod 19)
\end{aligned}$\\
នោះ $f(1)\vdots 19$\quad(ពិត)\\
ឧបមាវាពិតដល់ $n=k$\\
$\Rightarrow f(k)=2^{2^{6k}}+3\vdots 19\\
2^{2^{6k}}=19t-3\quad ,t>0$\\
ស្រាយថាវាពិតដល់ $n=k+1\Rightarrow f(k+1)\vdots 19$\\
$\begin{aligned}[t]
\text{គេបាន} &f(k+1)=2^{2^{6(k+1)+2}}+3=2^{2^{6k+6+2}}+3=2^{2^{6k+2}\cdot 2^6}+3\\
&=\left(2^{2^{6k+2}}\right)^{64}+3=(19t-3)^{64}+3\\
&\equiv 3^{64}+3\equiv 0\quad (mod 19)
\end{aligned}$\\
នោះ $f(n)\vdots 19$\quad (ពិត)\\
\fbox{\so $2^{2^{6n+2}}+3\vdots 19$}
\item $2^{2^{2n+1}}+3 \vdots 7$
តាង $P(n)=2^{2^{2n+1}}+3$\\
បើ $n=1$គេបាន\\
$\begin{aligned}[t]
P(1)&=2^{2^3}+3=2^8+3=256+3=259\\
&\equiv 0\quad(mod7)
\end{aligned}$\\
$\Rightarrow P(1)\vdots 7$\quad (ពិត)\\
ឧបមាវាពិតដល់ $n=K$\\
$\Rightarrow P(k)=2^{2^{2k+1}}+3\vdots 7\\
2^{2^{2k+1}}=7t-3\quad ;t>0$\\
ស្រាយវាពិតដល់ $n=k+1$គេបាន\\
$\begin{aligned}[t]
P(k+1)&=2^{2^{2(k+1)+1}}+3=2^{2^{2k+2+1}}+3\\
&=2^{2^{2k+1}\cdot 2^2}+3=\left(2^{2^{2k+1}}\right)^4+3=(7t-3)^4+3\\
&\equiv 3^4+3\equiv 81+3\quad(mod 7)\\
&\equiv 84\equiv 0\quad(mod7)
\end{aligned}$\\
នោះ $P(n)\vdots 7$\quad(ពិត)\\
\fbox{\so $2^{2^{2n+1}}+3 \vdots 7$}
\end{enumerate}
\end{proof}
\begin{exercise}
 ដោយប្រើវិចារអនុមានរួមស្រាយបញ្ជាក់ថាចំពោះ$n\in{\N}$និង$k\in{\N}$,$k$សេសេ  គេបាន$k^{2^{n}}-1\vdots 2^{n+2}$។
\end{exercise}
\begin{proof}[\solution]
ស្រាយបញ្ជាក់ថា$k^{2^{n}}-1\vdots 2^{n+2}$\\
តាង​$F(n)=k^{2^{n}}-1$\\
បើ $n=1$គេបាន\\
$F(1)=k^{2^1}-1\\
=k^2-1=(k-1)(k+1)$\\
ដោយ​$k$សេស\\
តាង​$k=2t+1\quad t\in{\N}$គេបាន\\
$\begin{aligned}[t]
F(1)&=(2t+1-1)(2t+1+1)\\
&=2t(2t+2)=4t(t+1)
\end{aligned}$\\
ដោយ​$t(t+1)\vdots 2$\\
$\Rightarrow F(1)\vdots 2^{1+2}$\quad (ពិត)\\
ឧបមាវាពិតដល់ $n=q\quad q>0$\\
$\Rightarrow F(q)=k^{2^{q}}-1\vdots 2^{q+2}\\
k^{2^q}=2^{q+2}\times p+1$\\
ស្រាយវាពិតដល់ $n=q+1$គេបាន\\
$\begin{aligned}[t]
F(q+1)&=k^{2^{q+1}}-1=k^{2^q\cdot2}-1\\
&=\left(k^{2^q}\right)^2-1=\left(2^{q+2}\times p+1\right)^2-1\end{aligned}$\\
$\begin{aligned}[t]F(q+1)&=\left(2^{q+2}\right)^2\times p^2+2\cdot2^{q+2}\cdot p+1-1\\
&=2^{2q+4}\cdot p^2+2^{q+3}\cdot p=2^{q+3}\left(2^{q+1}p^2+p\right)\vdots 2^{q+3}\end{aligned}$\\
$\Rightarrow F(q+1)\vdots 2^{q+3}$\quad (ពិត)\\
\fbox{\so $k^{2^{n}}-1\vdots 2^{n+2}$}
\end{proof}
\begin{exercise}
ដោយប្រើវិចារអនុមានរួមស្រាយបញ្ជាក់ថា$\forall n \in n$
\begin{enumerate}[a]
\item $11^{n+2}+12^{2n+1}\vdots 133$
\item $6^{2n+1}+5^{n+2}\vdots 31$
\end{enumerate}
\end{exercise}
\begin{proof}[\solution]
ស្រាយបញ្ជាក់ថា$\forall n \in n$
\begin{enumerate}[a]
\item $11^{n+2}+12^{2n+1}\vdots 133$
តាង​$g(n)=11^{n+2}+12^{2n+1}$\\
បើ$n=1\Rightarrow g(1)\vdots 133\\
g(1)=11^3+12^3=(11+12)(11^2-11\cdot 12+12^2)\\
23(121-132+144)=23\times 133\vdots 133$\\
$\Rightarrow g(1)\vdots 133$\quad(ពិត)\\
ឧបមាវាពិតដល់ $n=k\quad ,k>0$គេបាន\\
$g(k)=11^{k+2}+12^{2k+1}\vdots 133\\
12^{2k+1}=133t-11^{k+2}$\\
ស្រាយវាពិតដល់ $n=k+1$គេបាន\\
$\begin{aligned}[t] 
g(k+1)&=11^{k+1+2}+12^{2(k+1)+1}=11\cdot 11^{k+2}+144\cdot 12^{2k+1}\\
&=11\cdot 11^{k+2}+144\left(133t-11^{k+2}\right)=11\cdot 11^{k+2}+144\cdot 133t-144\cdot 11^{k+2}\\
&=144\cdot 133t-133\cdot 11^{k+2}=133\left(144t-11^{k+2}\right)\vdots 133
\end{aligned}$\\
នោះ $g(n)\vdots 133$\quad (ពិត)\\
\fbox{\so $11^{n+2}+12^{2n+1}\vdots 133$}
\item $6^{2n+1}+5^{n+2}\vdots 31$
តាង $P(n)=6^{2n+1}+5^{n+2}$\\
បើ $n=1\Rightarrow P(1)\vdots 133$គេបាន\\
$\begin{aligned}[t]
P(1)&=6^3+5^3=216+125\\
&=341\vdots 31
\end{aligned}$\\
ឧបមាវាពិតដល់$n=k\quad,k>0$គេបាន\\
$P(k)=6^{2k+1}+5^{k+2}\vdots 31\\
6^{2k+1}=31t-5^{k+2}$\\
ស្រាយថាវាពិតដល់$n=k+1$គេបាន\\
$\begin{aligned}[t]
P(k+1)&=6^{2(k+1)+1}+5^{k+3}=6^{2k+2+1}+5^{k+3}\\
&=36\cdot 6^{2k+1}+5\cdot 5^{k+2}=36\left(31t-5^{k+2}\right)+5\cdot 5^{k+2}\\
&=36\cdot 31t-36\cdot 5^{k+2}+5\cdot 5^{k+2}=36\cdot 31t-31\cdot 5^{k+2}\vdots 31\\
\end{aligned}$\\
$\Rightarrow P(n)\vdots 31\quad$(ពិត)\\
\fbox{\so $6^{2n+1}+5^{n+2}\vdots 31$}
\end{enumerate}
\end{proof}
\begin{exercise}
ស្រាយបញ្ជាក់ថា$x+\frac{1}{2}\in{\Z}$នោះ$x^n+\frac{1}{x^n}\in{\Z}$ចំពោះ$x\in{\N}$។. 
\end{exercise}
\begin{proof}[\underline{\kv ដំណោះស្រាយ}]
ស្រាយបញ្ជាក់ថា​$x^n+\frac{1}{x^n}\in{\Z}$ចំពោះ$x\in{\N}$\\
បើ​$n=1\Rightarrow x+\frac{1}{x}\in {\Z}$\\
ឧបមាវាពិតដល់ $n=k\Rightarrow x^k+\frac{1}{x^k}\in {\Z}$\\
ស្រាយវាពិតដល់$n=k+1$គេបាន\\
$x^{k+1}+\frac{1}{x^{k+1}}\in \Z$\\
ដោយ​​$\left(x+\frac{1}{x}\right)\left(x^k+\frac{1}{x^k}\right)\in \Z\\
\left(x^{k+1}+\frac{1}{x^{k-1}}+x^{k-1}+\frac{1}{x^{k+1}}\right)\in \Z$\\
$\left(x^{k+1}+\frac{1}{x^{k+1}} \right)+ \left(x^{k-1}+\frac{1}{x^{k-1}}\right)\in \Z$\\
ដោយដឹងថា $\left(x^{k-1}+\frac{1}{x^{k-1}}\right)\in \Z$\\
នោះ $x^{k+1}+\frac{1}{x^{x+1}}\in \Z$\quad (ពិត)\\
\fbox{\so $x^{k+1}+\frac{1}{x^{x+1}}\in \Z\quad , \forall n\in \N$}
\end{proof}
\begin{exercise}
ស្រាយបញ្ជាក់ថា$1^{2002}+2^{2002}+3^{2002}+\cdots+2002^{2002} \vdots 11$
\end{exercise}
\begin{proof}[\solution]
ស្រាយបញ្ជាក់ថា$1^{2002}+2^{2002}+3^{2002}+\cdots+2002^{2002} \vdots 11$\\
តាង​$A=1^{2002}+2^{2002}+3^{2002}+\cdots+2002^{2002}$\\
យក​$a\in \N , gcd(a,11)=1$\\
តាមទ្រឹស្តីបទ $Fermat\quad a^{11-1}\equiv 1(mod 11)$\\
$\Rightarrow a^{2002}=\left(a^{10}\right)^{200}\times a^2\equiv a^2(mod 11)
$\\
យក​$a=1,2,3\cdots ,2002$\\
$
\bullet\text{បើ}a=1\Rightarrow 1^{2002}\equiv 1^{2}\quad (mod11)\\
\bullet\text{បើ}a=2\Rightarrow 2^{2002}\equiv 2^2\quad(mod11)\\
\bullet\text{បើ}a=3\Rightarrow 3^{2002}\equiv 3^2\quad(mod11)\\
\cdot\cdots\cdots\cdots\cdots\cdots\cdots\cdots\cdots\cdots\cdots\\
\bullet\text{បើ}a=2002\Rightarrow 2002^{2002}\equiv 200^2(mod11)
$\\
យកអង្គបូក​នឹងអង្គគេបាន\\
 $\begin{aligned}[t] A&\equiv 1^2+2^2+3^2+\cdots+2002^2
 \equiv \frac{1}{6}\times 2002(2002+1)(2\times2002+1)\quad(mod 11)\\
 &\equiv \frac{1}{6}\times 2002\times 2003\times 4005
 \equiv 1001\times 2003\times 1335\quad(mod 11)\\
 &\equiv 7\times 11\times 13\times 2003\times 1335\equiv 0\quad(mod 11)
\end{aligned}$\\
 \fbox{\so $1^{2002}+2^{2002}+3^{2002}+\cdots+2002^{2002} \vdots 11$}
\end{proof}
\begin{exercise}
ស្រាយបញ្ជាក់ថា$3^{2^{4n+1}}+2^{3^{4n+1}}+5 \vdots 22$ចំពោះ$\forall n \in {\N}$។
\end{exercise}
\begin{proof}[\solution]
ស្រាយបញ្ជាក់ថា$3^{2^{4n+1}}+2^{3^{4n+1}}+5 \vdots 22$ចំពោះ$\forall n \in {\N}$\\
តាង $B=3^{2^{4n+1}}+2^{3^{4n+1}}+5$\\
ដោយ​$22=2\times 11$\\
$\begin{aligned}[t]
\Rightarrow B&=(2+1)^{2^{4n+1}}+2^{3^{4n+1}}+(4+1)\\
&\equiv 2\equiv 0\quad(mod11)
\end{aligned}$\\
$\Rightarrow B\vdots 2$\\
តាមទ្រឹស្តីបទ$Fermat$\\
ដោយ​$gcd(3,11)=1\quad ,gcd(2,11)=1\text{គេបាន}\\
3^{11-1}\equiv 1(mod11)\\
2^{11-1}\equiv 1(mod11) $\\
$\begin{aligned}[t]
\text{នោះ} 2^{4n+1}&=2\left(2^4\right)^n=2(10+6)^n\\
&\equiv 2\times 6^n\equiv 2\quad (mod10)\\
\end{aligned}$\\
$2^{4n+1}=10k+2$\\
យក​$3^{4n+1}$ចែកនឹង$10$គេបាន\\
$\begin{aligned}[t]
3^{4n+1}&=3\left(3^2\right)^{2n}=3(10-1)^{2n}\\
&\equiv 3\quad (mod10)
\end{aligned}$\\
$\Rightarrow 3^{4n+1}=10t+3$\\
$\begin{aligned}[t]
\text{គេបាន} B&=3^{2^{4n+1}}+2^{3^{4n+1}}+5=3^{10k+2}+2^{10t+3}+5=3^2\left(306{10}\right)+2^3\left(2^{10}\right)^k+5\\
&\equiv 9+8+5\equiv 22\equiv 0\quad (mod11)
\end{aligned}$\\
នាំឱ្យ​$B\vdots 11$\quad (ពិត)\\
\fbox{\so $3^{2^{4n+1}}+2^{3^{4n+1}}+5 \vdots 22$ចំពោះ$\forall n \in {\N}$}
\end{proof}
\begin{exercise}
ស្រាយបញ្ជាក់ថា$\forall n \in{\N}$
\begin{enumerate}[a]
\item $2^{2^{6n+2}}+3 \vdots 19$
\item $2^{2^{2n+1}}+3 \vdots 7$
\item $2^{2^{4n+1}}+7 \vdots 11$ ។
\end{enumerate}
\end{exercise}
\begin{proof}[\solution]
ស្រាយបញ្ជាក់ថា$\forall n \in{\N}$
\begin{enumerate}[a]
\item $2^{2^{6n+2}}+3 \vdots 19$\\
ដោយ​$gcd(2,19)=1$\\
តាមទ្រឹស្តីបទ$Fermat$\\
$2^{19-1}\equiv 1\quad (mod 19)$\\
យក​​​ $2^{6n+2}$ចែកនឹង​$18$គេបាន\\
$\begin{aligned}[t]
2^{6n+2}&=2^2\cdot 2^{6n}
=4 \left( 2^3\right)^{2n}=4\left(9-1\right)^{2n}\\
&\equiv 4\quad (mod 18)
\end{aligned}$\\
$\Rightarrow 2^{6n+2}=18k+4$\\
$\begin{aligned}[t]
\Leftrightarrow &2^{2^{6n+2}}+3=2^{18k+4}+3=2^4\cdot 2^{18k}+3\\
&\equiv 19\equiv 0\quad(mod19)
\end{aligned}$\\
នាំឱ្យ​$2^{2^{6n+2}}+3 \vdots 19$\quad (ពិត)\\
\fbox{\so $2^{2^{6n+2}}+3 \vdots 19 $}
\item $2^{2^{2n+1}}+3 \vdots 7$\\
ដោយ​$gcd(2,7)=1$\\
តាមទ្រឹស្តីបទ$Fermat\\
2^6\equiv 1\quad (mod7)$\\
យក​$2^{2n+1}$ចែកនឹង​$6$\\
$\begin{aligned}[t]
\text{គេបាន} 2^{2n+1}&=2\cdot\left(2^2\right)^n=2(3+1)^n\\
&\equiv 2\quad (mod6)
\end{aligned}$\\
$\Rightarrow 2^{2n+1}=6t+2\quad ,t\in{\R}$\\
$\text{គេបាន​} {2^{2n+1}}+3=2^{6t+2}+2=2^{6t}\cdot 2^2+3\\
\equiv 4+3\equiv 0\quad (mod7)$\\
\fbox{\so $2^{2^{2n+1}}+3 \vdots 7$}
\item $2^{2^{4n+1}}+7 \vdots 11$
ដោយ​$gcd(2,11)=1$\\
តាមទ្រឹស្តីបទ$Fermat\\
\text{គេបាន} 2^{10}\equiv 1\quad(mod11)$\\
យក​$2^{4n+1}$ចែកនឹង​$10$\\
$\begin{aligned}[t]
\text{គេបាន} 2^{4n+1}&=2\left(10+6\right)^n\\
&\equiv 2\times 6^n\equiv  2\times 6\equiv 2\quad(mod10)
\end{aligned}$\\
$\Rightarrow 2^{4n+1}=10k+2\quad ,k\in \R$\\
$\begin{aligned}[t]
\text{គេបាន} 2^{2^{4n+1}}+7&=2^{10k+2}+7=2^{10k}\cdot 2^2+7=2^{10k}\cdot 4+7\\
&\equiv 4+7\equiv 0\quad (mod11)
\end{aligned}$\\
\fbox{\so $2^{2^{4n+1}}+7 \vdots 11$}
\end{enumerate}
\end{proof}
\begin{exercise}
ស្រាយញ្ជាក់ថា$a_1+a_2+\cdots+a_{2018} \vdots 30$នោះ
${a_1}^5+{a_2}^5+\cdots+a_{2018}^5 \vdots 30$ចំពោះ${a_i}\in{\N} \\,i=1,2,\cdots,2018$។
\end{exercise}
\begin{proof}[\solution]
ស្រាយថា​${a_1}^5+{a_2}^5+\cdots+a_{2018}^5 \vdots 30$\\
$\begin{aligned}[t]
\text{តាង}& M=a_1+a_2+\cdots+a_{2018}\\
&N={a_1}^5+{a_2}^5+\cdots+a_{2018}^5
\end{aligned}$\\
ស្រាយថាបើ​$M\vdots 30$នោះ​$N\vdots 30$\\
$\begin{aligned}[t]
\text{សង្កេត} &a_i^5-a_i=a_i\left(a_i^4-1\right)=a_i(a_i^2-1)(a_i^2+1)\\
&=a_i(a_i-1)(a_i+1)(a_i^2+1)
\end{aligned}​​ $\\
ដោយ​ $a_i(a_i-1)(a_i+1)\vdots 3!\vdots 6\\
\Rightarrow a_i^5-a_i\vdots 6$\\
ចំពោះ​$i=1,2,\cdots,2018$គេបាន\\
$
\bullet\text{បើ}i=1\Rightarrow a_1^5-a_1\vdots 6\\
\bullet\text{បើ}​i=2\Rightarrow a_2^5-a_2\vdots 6\\
\cdots\cdots\cdots\cdots\cdots\cdots\cdots\cdots\\
\bullet\text{បើ}i=2018\Rightarrow a_{2018}^5-a_{2018}\vdots 6
$\\
បូកអង្គនឹងអង្គគេបានៈ\\
$a_1^5-a_1+a_2^5-a_2+\cdots+a_{2018}^5-a_{2018}\vdots 6$\\
$\left(a_1^5+a_2^5+\cdots+a_{2018}^5\right)-\left(a_1+a_2+\cdots+a_{2018}\right)\vdots6\\
\Rightarrow N-M\vdots 6$
ដោយ​$M\vdots6$\quad (ព្រោះ$M\vdots30$)\\
នោះ​$N\vdots6$\quad $(1)$\\
តាមទ្រឹស្តីបទ$Fermat\quad ,a_i\in \N$\\
$a_i^5\equiv a_i\quad(mod5)\\
a_i^5-a_i\equiv 0\quad(mod5)$\\
បូកអង្គនឹងអង្គ​$N-M\vdots6$\\
ដោយ​$M\vdots5$(ព្រោះ$M\vdots30$)\\
នោះ​$N\vdots5\quad(2)$\\
តាម​$(1)$និង​$(2)$គេបាន​$N\vdots5\quad$(ពិត)\\
\fbox{\so ​${a_1}^5+{a_2}^5+\cdots+a_{2018}^5 \vdots 30$}
\end{proof}
\begin{exercise}
 ស្រាយបញ្ជាក់ថា$2^{70}+3^{70} \vdots 13$។
\end{exercise} 
\begin{proof}[\solution]
 ស្រាយបញ្ជាក់ថា$2^{70}+3^{70} \vdots 13$\\
 ដោយ​$gcd(2,13)=1 ,gcd(3,13)=1$\\
 តាមទ្រឹស្តីបទ$Fermat$គេបាន\\
 $2^{12}\equiv 1\quad (mod13)\\
 3^{12}\equiv 1\quad(mod13)$\\
$\begin{aligned}[t]
 \Rightarrow 2^{70}+3^{70}&=\left(2^{12}\right)^5\times 2^{10}+\left(3^{12}\right)^5\times 3^{10}\\
 &\equiv 2^{10}+3^{10}\equiv \left(2^2\right)^5+\left(3^2\right)^5\quad(mod13)
 \end{aligned}$\\
 តាមរូបមន្ត​ $a^n+b^n\vdots a+b$\;បើ​$n$សេស\\
 $\begin{aligned}[t]
 \text{គេបានៈ}&​2^{70}+3^{70}\equiv \left(2^2\right)^5+\left(3^2\right)^5\equiv 4^5+9^5\equiv (4+9)(4^4-4^3\cdot9+\cdots+9^4)\quad(mod13)\\
 &\equiv 13(4^4-4^3\cdot9+\cdots+9^4)\equiv 0\quad (mod13)
 \end{aligned}$\\
 \fbox{\so $2^{70}+3^{70}\vdots 13$}
\end{proof}
\begin{exercise}
 $p$ជាចំនួនបឋមធំជាង$7$។ ស្រាយថា$3^p-2^p-2 \vdots 42p$។
\end{exercise}
\begin{proof}[\solution]
ស្រាយថា$3^p-2^p-2 \vdots 42p$\\
តាង​$A=3^p-2^p-2 $\\
ដោយ​​ $42p=2\times 3\times 7\times p$\\
$\ast$បង្ហាញថា​$A\vdots2$\\
$\begin{aligned}[t]
\text{គេបាន}A&=3^p-2^p-1=(2+1)^p-2^p-1\\
&\equiv 1-0-1\equiv 0\quad(mod2)
\end{aligned}$\\
$\Rightarrow A\vdots2\qquad(1)$\\
$\ast$បង្ហាញ$A\vdots3$\\
$\begin{aligned}[t]
\text{គេបាន}​A&=3^p-2^p-1=3^p-(3-1)^p-1\\
&\equiv 0-0+1-1\equiv 0\quad(mod3)
\end{aligned}$\\
$\Rightarrow A\vdots 3\qquad(2)$\\
$\ast$បង្ហាញថា​$A\vdots7$\\
ដោយ​$gcd(3,7)=1\quad , gcd(2,7)=1$\\
តាមទ្រឹស្តីបទ​$Fermat$\\
$3^6\equiv 1\quad(mod7)\\
2^6\equiv 1\quad(mod7)$\\
ដោយ​$p$ជាចំនួនបឋមធំជាង​$7$\\
តាង$p=6k+5,6k+1\\
A=3^p-2^p-1$\\
$\begin{aligned}[t]
\Leftrightarrow A&=3^{6k+5}-2^{6k+5}-1\\
&\equiv 3^5-2^5-1\equiv 243-32-1\quad(mod7)\\
&\equiv 210\equiv 0\quad(mod7)\\
\end{aligned}$\\
$\begin{aligned}[t]
\Leftrightarrow A&=3^{6k+1}-2^{6k+1}-1\\
&\equiv 3-2-1\equiv 0\quad(mod7)\\
\end{aligned}$\\
$\Rightarrow A\vdots 7\quad(3)$\\
$\ast$បង្ហាញថា​$A\vdots p$\\
តាមទ្រឹស្តីបទ​$Fermat$\\
$-\underline{\begin{cases}
3^p\equiv 3\\
2^p\equiv 2
\end{cases}}$\\
$\Rightarrow 3^p-2^p\equiv 1\quad(modp)\\
3^p-2^p-1\equiv 1\quad(modp)$\\
នោះ$A\vdots p\qquad(4)$\\
តាម$(1),(2),(3)$និង$(4)$គេបាន\\
$A\vdots42p$\\
\fbox{\so $3^p-2^p-1\vdots 42p$}
\end{proof}
\begin{exercise}
 រកសំណល់នៃវិធីចែក$2008^{2018}$នឹង$13$។
\end{exercise}
\begin{proof}[\solution]
 រកសំណល់នៃវិធីចែក$2008^{2018}$នឹង$13$\\
 ដោយ$2008=13\times154+6$\\
 $\begin{aligned}[t]
 \Rightarrow 2008^{2018}&=(13\times 154+6)^{2018}\\
 &\equiv 6^{2018}\quad(mod13)
 \end{aligned}$\\
 ដោយ$ gcd(6,13)=1$\\
 តាបទ្រឹស្តីបទ$Fermat$គេបាន\\
 $6^{12}\equiv 1\quad(mod13)$\\
 យក​$2018$ចែកនឹង$12$\\
 $2018=12\times 168+2$\\
 $\begin{aligned}[t]
 \Rightarrow 6^{2108}&=6^{12\times168+2}\equiv 6^2\quad(mod13)\\
 &\equiv 36\equiv 10\quad(mod13)\\
 \end{aligned}$\\
 \fbox{\so សំណល់នៃវិធីចែក$2008^{2018}$នឹង$13$គឺ$10$}
\end{proof}
\begin{exercise}
រកសំណល់នៃវិធីចែក${3^{{2^{2018}}}}$នឹង$11$ដោយ$gcd(3,11)=1$។
\end{exercise}
\begin{proof}[\solution]
រកសំណល់នៃវិធីចែក${3^{{2^{2018}}}}$នឹង$11$\\
ដោយ​$gcd(3,11)=1$គេបានៈ\\
$3^{10}\equiv 1\quad(mod11)$\\
យក$2^{2018}$ចែកនឹង$10$\\
$\begin{aligned}[t]
2^{2018}&=2^{4\times 504+2}=\left(2^4\right)^{504}\times 2^2=(10+6)^{504}\times 4\\
&\equiv 6^{504}\times 4\quad(mod10)
 \;
 \begin{tikzpicture}
 \node [rectangle callout,draw=orange!70,fill=orange!70,callout absolute pointer={(0,1)}
] at (5,1) {$6^k$ចែកនឹង$10$បានសំណល់$6$};
 \end{tikzpicture}\\
&\equiv 6\times 4\equiv 24\equiv 4\quad(mod10)
\end{aligned}$\\
$\Rightarrow 2^{2018}=10k+4$\\
$\begin{aligned}[t]
\text{នោះគេបាន}3^{2^{2018}}&=3^{10k+4}\\
&\equiv 3^4\equiv 81\quad(mod11)\end{aligned}$\\
$3^{2^{2018}}\equiv 4\quad(mod11)
$\\
\fbox{\so សំណល់នៃវិធីចែក$3^{2^{2018}}$នឹង$11$គឺ$4$}
\end{proof}
\begin{exercise}
 រកសំណល់នៃវិធីចែក$4362^{4362}$ ចែកនឹង$11$។
\end{exercise}
\begin{proof}[\solution]
 រកសំណល់នៃវិធីចែក$4362^{4362}$ ចែកនឹង$11$\\
 គេបាន$4362=11\times 396+6$\\
 $\begin{aligned}[t]
 \Rightarrow 4362^{43762}&=(11\times396+6)^{4362}\\
 &\equiv 6^{4362}\quad(mod11)
 \end{aligned}$\\
 ដោយ$gcd(6,11)=1$\\
 តាបទ្រឹស្តីបទ$Fermat$\\
$6^{10}\equiv 1\quad(mod11)$\\
យក$54362$ចែកនឹង$10$គេបានៈ\\
$\begin{aligned}[t]
4362&=10\times 436+2\\
&\equiv 2\quad(mod10)\\
\end{aligned}$\\
$\Rightarrow 4362=10t+2$\\
$\begin{aligned}[t]
\text{នាំឱ្យ}\;6^{4362}&=6^{10t+2}=6^{10t}\cdot 6^2\\
&\equiv 36\equiv 3\quad(mod13)
\end{aligned}$\\
\fbox{\so សំណល់នៃវិធីចែក$4362^{4362}$នឹង$11$គឺ$3$}
\end{proof}
\begin{exercise}
 រកចំនួនគត់មានលេខពីរខ្ទង់ដែលពេលចែកនឹង$5$មានសំណល់$2$ចែកនឹង$9$មានសំណល់$7$។
\end{exercise}
\begin{proof}[\solution]
 រកចំនួនគត់មានលេខពីរខ្ទង់\\
 តាង$N$ជាចំនួនមានលេខពីរខ្ទង់\\
 $\begin{aligned}[t]
 \text{ដោយ}& N\equiv 2\quad(mod5)\\
 &N\equiv 9\quad(mod 7)
 \end{aligned}$\\
 តាង​$N=r_o\quad(mod 9\times 5)$\\
 យក$r_o=9t+7\quad , t=0,1,2,3,4$\\
 មានតែ$t=0\Rightarrow r_o=7\equiv 7\quad(mod 5)$\\
 $\Rightarrow N\equiv 7\quad(mod 45)\\
 \Leftrightarrow N\equiv 45\times k+7$\\
 $\divideontimes$បើ​ $k=1\\
 \Rightarrow N=45\times 1+7=52$
  \;
 \begin{tikzpicture}
 \node [rectangle callout,draw=black,text=black ,callout absolute pointer={(0,1)}
] at (5,1) {52ចែកនឹង5សំណល់2ហើយចែកនឹង9សំណល់7};
  \end{tikzpicture}\\
 $\divideontimes$បើ$k=2\\
 \Rightarrow N=45\times 2+7=97$
 \;
 \begin{tikzpicture}
 \node [rectangle callout,draw=black,text=black,callout absolute pointer={(0,-0.5)}
] at (5,1) {97ចែកនឹង5សំណល់2ហើយចែកនឹង9សំណល់7};
 \end{tikzpicture}\\
 \fbox{\so ចំនួនមានលេខពីរខ្ទង់គឺ$57$និង$97$}
\end{proof}
\begin{exercise}
 រកសំណល់នៃវិធីចែក$2^{2017}$នឹង$35$។
\end{exercise}
\begin{proof}[\solution]
 រកសំណល់នៃវិធីចែក$2^{2017}$នឹង$35$\\
 ដោយ$35=5\times7\\
 gcd(5,7)=1$\\
 $\begin{aligned}[t]
 \text{គេបាន} 2^{2017}&=2^{2016}\cdot 2=\left(2^2\right)^{1008}\cdot 2 =4^{1008}\cdot 2=(5-1)^{1008}\cdot 2\\
 &\equiv 2\quad(mod5)
 \end{aligned}$\\
 $\begin{aligned}[t]
 2^{2017}&=\left(2^3\right)^{672}\cdot2=(7+1)^{672}\cdot2\\
 &\equiv 2\quad(mod7)
 \end{aligned}$\\
 \fbox{\so សំណល់នៃវិធីចែក$2^{2017}$នឹង$35$គឺ$2$}
\end{proof}
\begin{exercise}
 រកសំណល់នៃវិធីចែក$3^{2^{4n+1}}$ចែកនឹង$65$។
\end{exercise}
\begin{proof}[\solution]
 រកសំណល់នៃវិធីចែក$3^{2^{4n+1}}$ចែកនឹង$65$\\
 ដោយ​\;$55=5\times 11$\\
 ដែល\;$gcd(3,5)=1\quad ,gcd(3,11)=1$
 តាមទ្រឹស្តីបទ$Fermat$\\
 $\begin{cases}
 3^4\equiv 1\quad(mod5)\\
 3^{10}\equiv 1\quad(mod11)
 \end{cases}$\\
 យក$2^{4n+1}$ចែកនឹង$4$គេបានៈ\\
 $\begin{aligned}[t]
 2^{4n+1}&=2\left(2^2\right)^{2n}\\
 &\equiv 0\quad(mod4)
 \end{aligned}$\\
 យក$2^{4n+1}$ចែកនឹង$10$គេបានៈ\\
 $2^{4n+1}=\left(2^4\right)^{n}\cdot2=(11+6)^n\cdot2$\\
 $\begin{aligned}[t]
 2^{4n+1}&\equiv 6^n\cdot2\quad(mod10)\\
 &\equiv 6\times 2\equiv 2\quad(mod10)
\end{aligned}$\\
$\Rightarrow\begin{cases} 2^{4n+1}=4k\\
 2^{4n+1}=10k+2
 \end{cases}$\\
 គេបាន\\
$3^{2^{4n+1}}=3^{4k}\equiv 1\quad(mod5)$\\
ម្យ៉ាងទៀត\\
$\begin{aligned}[t]
3^{2^{4n+1}}&=3^{10k+2}=9\cdot 3^{10k}\\
&\equiv 9\quad(mod11)
\end{aligned}$\\
តាង$3^{2^{4n+1}}=r_o$\\
យក​\;$r_o=11t+9\quad ,t=0,1,2,3,4$\\
$\begin{aligned}[t]
\text{មានតែ}t=2\Rightarrow r_o&=11\times 2+9\\
&=31\equiv 1\quad(mod5)
\end{aligned}$\\
\fbox{\so សំណល់នៃវិធីចែក$3^{2^{4n+1}}$នឹង$55$គឺ$31$}
\end{proof}
\begin{exercise}
 រកសំណល់នៃវិធីចែក
\begin{enumerate}[a]
\item $3^{100}$នឹង$25$
\item $2^{2007}$នឹង$49$។
\end{enumerate}
\end{exercise}
\begin{proof}[\solution]
រកសំណល់នៃវិធីចែក
\begin{enumerate}[a]
\item $3^{100}$នឹង$25$\\
យើងមាន$p^2=25\Rightarrow p=5$\\
$p(p-1)=5\times 4=20$\\
ដោយ$gcd(3,5)=1\\
\Rightarrow 3^{20}\equiv 1\quad(mod5^2)$\\
នោះ$3^{1000}=3^{20\times 5}\equiv 1\quad(mod25)$\\
\fbox{\so សំណល់នៃវិធីចែក$3^{1000}$នឹង$25$គឺ$1$}
\item $2^{2007}$នឹង$49$\\
យើងមាន$p^2=49\Rightarrow p=7\\
\Rightarrow p(p-1)=7\times 6=42$
ដោយ$gcd(2,7)=1\\
\Rightarrow 2^{42}\equiv 1\quad (mod7^2)$\\
$\begin{aligned}[t]
\text{គេបាន}\; &2^{2007}=2^{42\times 47+33}
=2^{42\times 47}\cdot 2^{33}\\
&\equiv 2^{33}\equiv \left(2^3\right)^{11}\quad\left( mod7^2\right)\\
&\equiv (7+1)^{11}\equiv 11\times 7+1\quad\left( mod7^2\right)
\;
 \begin{tikzpicture}
 \node [rectangle callout,draw=yellow!70,fill=yellow!70,callout absolute pointer={(0,1)}
] at (5,1) {\kml រូបមន្ត\;$(a+b)^n\equiv nab^{n-1}+b^n\quad\left( mod a^2\right)$};
 \end{tikzpicture}\\
&\equiv 78\equiv 49+29\quad\left( mod7^2\right)\\
&\equiv 29\quad\left( mod7^2\right)\end{aligned}$\\
\fbox{\so សំណល់នៃវិធីចែក$2^{2007}$នឹង$49$គឺ$29$}
\end{enumerate}
\end{proof}
\begin{exercise}
 រកសំណល់នៃវិធីចែក
\begin{enumerate}[a]
\item $3^{4^{15}}$នឹង$121$
\item $15^{15^{15}}$នឹង$49$។
\end{enumerate}
\end{exercise}
\begin{proof}[\solution]
 រកសំណល់នៃវិធីចែក
 \begin{enumerate}[a]
 \item $3^{4^{15}}$នឹង$121$\\
 ដោយ$p^2=121\Rightarrow p=11\\
 p(p-1)=11\times 10=110\\
 \text{ដោយ}\; gcd(3,11)=1$\\
 យក$4^{15}$ចែកនឹង$11$\\
 $\begin{aligned}[t]
 \text{គេបាន}\; 4^{15}&=2^{30}=\left(2^{5}\right)^6=32^6=(33-1)^6\\
 &\equiv 1\quad(mod11)
\end{aligned} $\\
យក$4^{15}$ចែកនឹង$10$\\
$\begin{aligned}[t]
\text{គេបាន}\;& 4^{15}=2^{30}=2^{4\times 7+2}
&4^{15}\equiv 6\times 4\equiv 4\quad(mod10)\end{aligned}$\\
តាង$4^{15}=r_o\quad(mod10\times 11)$\\
យក$r_o=11t+1\quad , t=0,1,2,\cdots,9$\\
$\begin{aligned}[t]
\text{ឱ្យ}t=3\Rightarrow r_o&=34\\
&\equiv 4\quad(mod10)
\end{aligned}$\\
គេបាន$4^{15}\equiv 34\quad(mod10\times 11)\\
4^{15}=10\times 11k+34$\\
$\begin{aligned}[t]
\text{គេបាន}\; 3^{4^{15}}&=3^{10\times11k+34}\\
&\equiv 3^{34}\equiv \left(3^5\right)^6\cdot 3^4\quad(mod121)\\
&\equiv (121\times 2+1)\times81\equiv 81\quad(mod121)\\
\end{aligned}$\\
\fbox{\so សំណល់នៃវិធីចែក$3^{4^{15}}$នឹង$121$គឺ$81$}
\item $15^{15^{15}}$នឹង$49$\\
គេបាន$p^2=49\Rightarrow p=7\\
p(p-1)=7\times 6$\\
ដោយ$gcd(15,7)=1$គេបាន\\
$15^{7\times 6}\equiv 1\quad(mod7^2)$\\
យក$15^{15}$ចែកនឹង$6$\\
$\begin{aligned}[t]
\text{គេបាន}\; 15^{15}&=(12+3)^{15}\\
&\equiv 3^{15}\equiv (3^3)^{5}\equiv (24+3)^5\quad(mod6)\\
&\equiv 3^5\equiv 243\equiv (6\times 40+3)\equiv 3\quad(mod6)
\end{aligned}$\\
យក$15^{15}$ចែកនឹង$7$\\
$\begin{aligned}[t]
\text{គេបាន}\; 15^{15}&=(14+1)^{15}\\
&\equiv 1\quad(mod7)
\end{aligned}$\\
តាង$15^{15}=r_o$\\
យក$r_o=7t+1\quad ,t=0,1,2,\cdots, 5$\\
$\begin{aligned}[t]
\text{ឱ្យ}t=2\Rightarrow r_o&=14+1=15\\
&\equiv 3\quad(mod6)
\end{aligned} $\\
$15^{15}\equiv 15\quad(mod6\times 7)\\
\Rightarrow 15^{15}=6\times7k+7$\\
$\begin{aligned}[t]
\text{នាំឱ្យ}15^{15^{15}}&=15^{6\times7k+7}=15^{6\times7k}\cdot15^7\\
&\equiv 15^7\equiv 15^{14}\times 15\equiv (14+1)^{15}\times15\quad(mod49)\\
&\equiv (7\times 2+1)^{15}\times15\equiv 1\times15\equiv 15\quad(mod49)
\end{aligned}$\\
\fbox{\so សំណល់នៃវីធីចែក$15^{15^{15}}$នឹង$49$គឺ$15$}
 \end{enumerate}
\end{proof}
\begin{exercise}
 រកសំណលក្នុងវិធីចែក
\begin{enumerate}[a]
\begin{multicols}{2}
\item $6^{1991}$នឹង$28$
\item $35^{150}$នឹង$425$
\item $35^{12^{11}}$នឹង$325$
\item $22^{2002}$នឹង$1001$។
\end{multicols} 
\end{enumerate}
\end{exercise}
\begin{proof}[\solution]
រកសំណលក្នុងវិធីចែក
\begin{enumerate}[a]
\item $6^{1991}$នឹង$28$\\
យើងមាន$m=28=2^2\times 7$\\
$\begin{aligned}[t]
\Rightarrow 6^{1991}&=(2\times3)^{1991}=2^{1991}\times3^{1991}\\
&=2^2\times 2^{1989}\times 3^{1989}\times3^2
&=2^2\times6^{1989}\times 9
\end{aligned}$\\
$\star$សម្រួល$2^2$រួចចែក$6^{1989}$នឹង$7$\\
$\begin{aligned}[t]
\text{គេបាន}\; 6^{1989}\times 9&=(7-1)^{1989}\times 9=(7-1)^{1989}\times (7+2)\\
&\equiv -2\equiv -7+5\equiv 5\quad(mod7)
\end{aligned}$\\
$\star$គុណអង្គទាំងពីេរនឹង$2^2$គេបាន\\
$6^{1991}\equiv 20\quad(mod28)$\\
\fbox{\so សំណល់នៃវិធីចែក$6^{1991}$នឹង$28$គឺ$20$}
\item $35^{150}$នឹង$425$\\
ដោយ​$m=425=5^2\times17$\\
$\begin{aligned}[t]
\text{គេបាន}\; &35^{150}=7^{150}\times5^{150}=7^2\times7^{148}\times5^2\times5^{148}\\
&=7^2\times35^{148}\times5^2=5^2\times 35^{148}\times49
\end{aligned}$\\
$\star$សម្រួល$5^2$រួចចែក$3^{148}\times 49$នឹង$17$\\
$\begin{aligned}[t]
\Rightarrow &35^{148}\times49=(34+1)^{148}\times(34+15)\\
&\equiv 1\times15\equiv 15\quad(mod17)
\end{aligned}$\\
$\star$គុណអង្គទាំងពីរនឹង$5^2$គេបានៈ\\
$35^{150}\equiv 15\times25\quad(mod425)\\
3^{150}\equiv 375\qquad(mod425)$\\
\fbox{\so សំណល់នៃវិធីចែក$35^{150}$នឹង$425$គឺ$375$}
\item $35^{12^{11}}$នឹង$325$\\
យើងមាន$m=325=5^2\times 13$\\
$\begin{aligned}[t]
35^{12^{11}}&=5^{12^{11}}\times7^{12^{11}}=5^2\times5^{12^{11}-2}\times7^{12^{11}-2}\times7^2\\
&=35^{12^{11}-2}\times49
\end{aligned}$\\
$\star$សម្រួល$5^2$រួចចែក$35^{12^{11}-2}\times49$នឹង$13$\\
$\begin{aligned}[t]
\text{គេបាន}\;35^{12^{11}-2}\times49&=(13\times3-4)^{12^{11}-2}\times (13\times3+10)\\
&\equiv 4^{12^{11}-1}\times10\quad(mod13)
\end{aligned}$\\
ដោយ$gcd(4,13)=1$\\
តាមទ្រឹស្តីបទ$Fermat\\
4^{12}\equiv 1\quad(mod13)$
យក$12^{11}-2$ចែកនឹង$12$\\
$\begin{aligned}[t]
\text{គេបាន} 12^{11}-2&\equiv -2\quad(mod12)\\
&\equiv 10\quad(mod12)
\end{aligned}$\\
$\Rightarrow 12^{11}-2=12k+10$\\
$\begin{aligned}[t]
\text{នាំឱ្យ}\; &4^{12^{11}-2}=4^{12k+10}\times 10\\
&\equiv 4^{10}\times 10\equiv (4^3)^3\times 4\times 10\quad(mod13)\\
&\equiv (64)^3\times 10\equiv (65-1)^3\times (39+1)\quad(mod13)\end{aligned}$\\
$4^{12^{11}-2}\equiv -1\equiv 12\quad(mod13)
$\\
$\Rightarrow 35^{12^{11}-2}\times 49\equiv 12\quad(mod13)$\\
គុណអង្គទាំងពីរនឹង$5^2$គេបាន\\
$35^{12^{11}}\equiv 300\quad(mod325)$\\
\fbox{\so សំណល់នៃវិធីចែក$35^{12^{11}}$នឹង$325$គឺ$300$}
\item $22^{2002}$នឹង$1001$\\
យើមាន $m=1001=7\times 11\times 13$\\
$\begin{aligned}[t]
\text{គេបាន}\; 22^{2002}&=2^{2002}\times 11^{2002}\\
&=2\times 2^{2001}\times 11\times 11^{2001}=2\times 22^{2001}\times 11
\end{aligned}$\\
$\ast$សម្រួល$11$រួចចែក$2\times 22^{2001}$នឹង$7\times 13$\\
$\looparrowright$រកសំណល់នៃវិធីចែក$2\times 22^{2001}$នឹង$7$\\
$\begin{aligned}[t]
\text{គេបាន}\; 2\times 22^{2001}&=2\times (21+1)^{2001}\\
&\equiv 2\quad(mod7)
\end{aligned}$\\
$\looparrowright$រកសំណល់នៃវិធីចែក$2\times 22^{2001}$នឹង$13$\\
$\begin{aligned}[t]
\text{គេបាន}\;& 2\times 22^{2001}=2\times (26-4)^{2001}\\
&\equiv -2\times 4^{2001}\equiv -2(2^3)^{667}\equiv -2(65-1)^{667}\equiv 2\quad(mod13)
\end{aligned}$\\
ដោយ$r_1=r_2=r_o=2$\\
គេបាន $2\times 22^{2001}\equiv 2\quad(mod7\times 13)$\\
$\ast$គុណអង្គទាំងពីរនឹង$11$វិញគេបាន\\
$11\times 2\times 22^{2001}\equiv 22\quad(mod1001)\\
22^{2002}\equiv 22\quad(mod1001)$\\
\fbox{\so សំណល់នៃវិធីចែក$22^{2002}$នឹង$1001$គឺ$22$}
\end{enumerate}
\end{proof}
\begin{exercise}
 រកសំណល់នៃ$3^{2019}$ចែកនឹង$35$។
\end{exercise}
\begin{proof}[\solution]
 រកសំណល់នៃ$3^{2019}$ចែកនឹង$35$\\
 ដោយ $35=7\times 5\quad , gcd(7,5)=1$\\
$\looparrowright$ រកសំណល់នៃវិធីចែក$3^{2019}$នឹង$5$\\
 $\begin{aligned}[t]
 \text{គេបាន}\; 3^{2019}&=(3^2)^{1009}\times3=(10-1)^{1009}\times 3\\
 &\equiv -3\equiv -5+2\equiv 2\quad(mod5) 
 \end{aligned}$\\
 $\looparrowright$ រកសំណល់នៃវិធីចែក$3^{2019}$នឹង$7$\\
 $\begin{aligned}[t]
 \text{គេបាន}\; 3^{2019}&=(3^3)^{673}=(27)^{673}=(28-1)^{673}\\
 &\equiv -1\equiv -7+6\equiv 6\quad(mod7)
 \end{aligned}$\\
 $\Rightarrow 3^{2019}=r_o$\\
 ដែល$r_o=7k+6\quad ,k=0,1,2,3,4$\\
 $\begin{aligned}[t]
 \text{យក}k=3\Rightarrow r_o&=7\times 3+6\\
 &=27\equiv 2\quad(mod5)
 \end{aligned}$\\
 $\Rightarrow 3^{2019}\equiv 27\quad(mod35)$\\
 \fbox{\so សំណល់នៃចែក$3^{2019}$នឹង$35$គឺ$27$}
\end{proof}
\begin{exercise}
រកសំណល់នៃ$3^{2018}$ចែកនឹង$49$។
\end{exercise}
\begin{proof}[\solution]
សំណល់នៃ$3^{2018}$ចែកនឹង$49$\\
យើងមាន$p^2=49\Rightarrow p=7\\
p(p-1)=7(7-1)=7\times 6\\
\Rightarrow 3^{7\times 6}=3^{42}\equiv 1\quad(mod49)$\\
គេបាន$3^{2018}=3^{7\times 6+2}\equiv 3^2\quad(mod49)$\\
\fbox{\so សំណល់នៃវិធីចែក$3^{2018}$នឹង$49$គឺ$9$}
\end{proof}
\begin{exercise}
ដោយប្រើទ្រឹស្តីបទ$Euler$រកសំណល់នៃវិធីចែក$3^{2019}$នឹង$28$។
\end{exercise}
\begin{proof}[\solution]
រកសំណល់នៃវិធីចែក$3^{2019}$នឹង$28$\\
ដោយ​$28=7\times 4=2^2\times 7$\\
$\begin{aligned}[t]
\text{រក}\varphi(28)&=28\left(1-\frac{1}{2}\right)\\
&=28\left(\frac{1}{2}\right)\left(\frac{6}{7}\right)=12
\end{aligned}$\\
តាមទ្រឹស្តីបទ {\en Euler}\\
$gcd(3,28)=1\\
3^{\varphi(28)}=3^{12}\equiv 1\quad(mod28)$\\
$\begin{aligned}[t]
\text{គេបាន}\; 3^{2019}&=3^{12\times 68+3}\\
&\equiv 3^3\equiv 27\quad(mod28)
\end{aligned}$\\
\fbox{\so សំណល់នៃវិធីចែក$3^{2019}$នឹង$28$គឺ$27$}
\end{proof}
\begin{exercise}
ដោយប្រើទ្រឹស្តីបទ{\en Euler}រកសំណល់ក្នុងវិធីចែក
\begin{enumerate}[a]
\item $109^{345}$នឹង$14$
\item $5^{70}+7^{50}$ចែកនឹង$12$។
\end{enumerate}
\end{exercise}
\begin{proof}[\solution]
រកសំណល់ក្នុងវិធីចែក
\begin{enumerate}[a]
\item $109^{345}$នឹង$14$\\
$\begin{aligned}[t]
\text{ដោយ}\; 109^{345}&=(14\times 7+11)^{345}\\
&\equiv 11^{345}
\end{aligned}$\\
$\begin{aligned}[t]
\text{រក}\; \varphi(14)&=\varphi(2)\times\varphi(7) \\
&=(2-1)(7-1)=6
\end{aligned}$\\
$\ast$តាមអនុគមន៍{\en Euler}\\
$gcd(11,14)=1\\
\Rightarrow 11^{\varphi(14)}=11^6\equiv 1\quad(mod14)\\
\text{គេបាន}\; 11^{6\times57+3}\equiv 11^3\quad(mod14)$\\
$\looparrowright$ចែក$11^3$នឹង$14$\\
$\begin{aligned}[t]
\text{គេបាន}\; 11^3&=(14-3)^{3}\\
&\equiv -3^3\equiv -27\quad(mod14)\\
&\equiv (-28+1)\equiv 1\quad(mod14)
\end{aligned}$\\
\fbox{\so សំណល់នៃវិធីចែក$109^{345}$នឹង$14$គឺ$1$}
\item $5^{70}+7^{50}$ចែកនឹង$12$\\
ដោយ$m=12=2^2\times3$\\
$\begin{aligned}[t]
\text{រក}\;\varphi(12)&=12\left(1-\frac{1}{2}\right)\left(1-\frac{1}{3}\right) \\
&=12\times \frac{1}{2}\times\frac{2}{3}=4
\end{aligned}$\\
$\ast$តាមទ្រឹស្តីបទ{\en Euler}\\
$gcd(5,12)=1\Rightarrow 5^{\varphi(12)}=5^4\equiv 1\quad(mod12)\\
gcd(7,12)=1\Rightarrow 7^{\varphi(12)}=7^4\equiv 1\quad(mod12)$\\
$\looparrowright$យក$5^{70}$និង$7^{50}$ចែកនឹង$12$\\
$\begin{aligned}[t]
\text{គេបាន}\; 5^{70}&=5^{4\times12+2}\\
&\equiv 25\equiv (12\times2+1)\equiv 1\quad(mod12)
\end{aligned}$\\
$\begin{aligned}[t]
\text{ម្យ៉ាងទៀត}\; 7^{50}&=7^{4\times12+2}\\
&\equiv 7^2\equiv 49\equiv(12\times4+1)\equiv 1\quad(mod12)
\end{aligned}$\\
$\Rightarrow 5^{70}+7^{50}\equiv 1+1\equiv 2\quad(mod12)$\\
\fbox{\so សំណល់នៃវិធីចែក$5^{70}+7^{50}$គឺ$2$}
\end{enumerate}
\end{proof}
\begin{exercise}
ដោយប្រើទ្រឹស្តីបទ{\en Euler}រកសំណល់នៃវិធីចែក
\begin{enumerate}[a]
\item $2008^{2008}$នឹង$35$
\item $11^{11^{11}}$នឹង$30$។
\end{enumerate}
\end{exercise}
\begin{proof}[\solution]
រកសំណល់នៃវិធីចែក
\begin{enumerate}[a]
\item $2008^{2008}$នឹង$35$\\
យើងមាន$m=35=5\times7$\\
$\begin{aligned}[t]
\text{រក}\; \varphi(35)&=35\left(1-\frac{1}{5}\right)\left(1-\frac{1}{7}\right)\\
&=35\left(\frac{4}{5}\right)\left(\frac{6}{7}\right)=24
\end{aligned}$\\
$\begin{aligned}[t]
\text{ដោយ}\; 2008^{2008}&=(35\times57+13)^{2008}\\
&\equiv 13^{2008}\quad(mod35)
\end{aligned}$\\
$\ast$តាមទ្រឹស្តីបទ{\en Euler}\\
ដោយ$gcd(13,35)=1\\
\Rightarrow 13^{\varphi(35)}=13^{24}\equiv 1\quad(mod35)$\\
$\begin{aligned}[t]
\text{គេបាន}\; &13^{2008}=13^{24\times83+16}\\
&\equiv 13^{16}\equiv (35\times4+29)^8\equiv 29^8\quad(mod35)\\
&\equiv (35-6)^8\equiv 6^8\quad(mod35)\end{aligned}$\\
$13^{2008}\equiv \left(6^2\right)^4\equiv (35+1)^4\equiv 1\quad(mod35)
$\\
\fbox{\so សំណល់នៃវិធីចែក$2008^{2008}$នឹង$35$គឺ$1$}
\item $11^{11^{11}}$នឹង$30$\\
យើងមាន$m=30=2\times3\times5$\\
$\begin{aligned}[t]
\text{រក}\varphi(30)&=\varphi(2)\times\varphi(3)\times\varphi(5)\\
&=(2-1)(3-1)(5-1)=8
\end{aligned}$\\
$\ast$តាមទ្រឹស្តីបទ{\en Euler}\\
ដោយ$gcd(11,30)=1\\
11^{\varphi}=11^8\equiv 1\quad(mod30)$\\
$\looparrowright$យក$11^{11}$ចែកនឹង$8$\\
$\begin{aligned}[t]
\text{គេបាន}\; 11^{11}&=(8+3)^{11}\\
&\equiv 3^{11}\equiv (3^2)^5\times3\equiv 3\quad(mod8)
\end{aligned}$\\
$\Rightarrow 11^{11}=8k+3$\\
$\begin{aligned}[t]
\text{នោះ}\; 11^{11^{11}}&=11^{8k+3}=11^{8k}\cdot11^3\\
&\equiv 11^3\equiv 121\times11\quad(mod30)\\
&\equiv (120+1)\times11\equiv 11\quad(mod30)
\end{aligned}$\\
\fbox{\so សំណល់នៃវិធីចែក$11^{11^{11}}$នឹង$30$គឺ$11$}
\end{enumerate}
\end{proof}
\begin{exercise}
ដោយប្រើ{\en Euler}រកសំណល់នៃវិធីចែក
\begin{enumerate}[a]
\item $3\times 5^{70}+4\times 7^{100}$នឹង$132$
\item $5^{70}+7^{50}$នឹង$12$។
\end{enumerate}
\end{exercise}
\begin{proof}[\solution]
រកសំណល់នៃវិធីចែក
\begin{enumerate}[a]
\item $3\times 5^{70}+4\times 7^{100}$នឹង$132$\\
$\begin{aligned}[t]
\text{រក}\varphi(132)&=\varphi(12\times11)\\
&=\varphi(2^2-1)\times\varphi(3)\times\varphi(11)\\
&=(2^2-2^1)(3-1)(11-1)=40
\end{aligned}$\\
តាមទ្រឹស្តីបទ{\en Euler}\\
$\begin{cases}
gcd(5,132)=1\Rightarrow5^{\varphi(132)}=5^{40}\equiv 1\quad(mod132)\\
gcd(7,132)=1\Rightarrow7^{\varphi(132)}=7^{132}\equiv 1\quad(mod132)\end{cases}\\
\begin{aligned}[t]
\Rightarrow&3\times5^{75}+4\times7^{100}=3\times5^{40+35}+4\times7^{2\times40+20}\\
&\equiv 3\times5^{35}+4\times7^{20}\quad(mod132)\end{aligned}$\\
$
\text{ដោយ}\; 5^{35}=(5^3)^{11}\times5^2=(132-7)^{11}\times5^2\\
\equiv -7^{11}\times5^2\equiv -(7^4)^2\times7^3\times25\quad(mod132)\\
\equiv -(132\times18+25)^2\times7^3\times 25\quad(mod132)\\
\equiv -(25)^2\times343\times25\equiv -43\quad(mod132)\\
\Rightarrow 3\times5^{35}\equiv -129\equiv -132+3\equiv 3\quad(mod132)
$\\
$\begin{aligned}
\text{ម្យ៉ាងទៀត}7^{20}&=(7^4)^5=(132\times18+25)^5\\
&\equiv 25^5\equiv 25(25)^4\equiv 25(132\times4+97)^2\quad(mod132)\\
&\equiv 25\times97^2\equiv 25(132-35)^2\quad(mod132)\\
&\equiv 25\times35^2\equiv (132\times232+1)\equiv 1\quad(mod132)
\end{aligned}$\\
$\text{គេបាន}\; 3\times5^{75}+4\times7^{100}\equiv 3+1\times4
\equiv 7\quad(mod132)$\\
\fbox{\so សំណល់នៃវិធីចែក $3\times 5^{70}+4\times 7^{100}$នឹង$132$គឺ$7$}
\item $5^{70}+7^{50}$នឹង$12$
$5^{70}+7^{50}$នឹង$12$\\
យើងមាន$m=12=2^2\times3$\\
$\text{រក}\; \varphi(12)=12\left(1-\frac{1}{2}\right)\left(1-\frac{1}{3}\right)=12\times \frac{1}{2}\times\frac{2}{3}$\\
$\Rightarrow \varphi(12)=4$\\
$\ast$តាមទ្រឹស្តីបទ{\en Euler}\\
$\begin{cases}
gcd(5,12)=1\Rightarrow5^{\varphi(12)}=5^4\equiv 1\quad(mod12)\\
gcd(7,12)=1\Rightarrow7^{\varphi(12)}=7^4\equiv 1\quad(mod12)
\end{cases}$\\
$\begin{aligned}[t]
\Rightarrow 5^{70}&=5^{4\times17+2}\\
&\equiv5^2\equiv (12\times2+1)\equiv1\quad(mod12) 
\end{aligned}$\\
$\begin{aligned}7^{50}&=7^{4\times12+2}\\
&\equiv 7^2\equiv (12\times4+1)\equiv 1\quad(mod12)
\end{aligned}$\\
$\Rightarrow 5^{70}+7^{50}\equiv 1+1\equiv 2\quad(mod12)$\\
\fbox{\so សំណល់នៃវិធីចែក $5^{70}+7^{50}$នឹង$12$គឺ$2$}
\end{enumerate}
\end{proof}
\begin{exercise}
 ស្រាយថា$0.3(2003^{2017}-2007^{2019})$ជាចំនួនគត់។
\end{exercise}
\begin{proof}[\solution]
ស្រាយថា$0.3(2003^{2017}-2007^{2019})$ជាចំនួនគត់\\
$\begin{aligned}
\text{តាង}\; A&=0.3\left(2003^{2017}-2007^{2019}\right)\\
&=\frac{3}{10}\left(2003^{2017}-2007^{2019}\right)
\end{aligned}$\\
$\begin{aligned}
\text{យើងមាន}\; &2003^{2017}=(2000+3)^{2017}\\
&\equiv 3^{2017}\equiv 3^{4\times504+1}\equiv 3\quad(mod10)
\end{aligned}$\\
$\begin{aligned}
\text{ម្យ៉ាងទៀត}\; &2007^{2019}=(2000+7)^{2019}\\
&\equiv 7^{2019}\equiv 7^{4\times504+3}\equiv 7^3\quad(mod10)\\
&\equiv 343\equiv 3\quad(mod10)
\end{aligned}$\\
$\Rightarrow 2003^{2017}-2007^{2019}\equiv 3-3\equiv 0\quad(mod10)$\\
នាំឱ្យ$A$ជាចំនួនគត់\quad(ពិត)\\
\fbox{\so $2003^{2017}-2007^{2019}$ជាចំនួនគត់}
\end{proof}
\begin{exercise}
ស្រាយបញ្ជាក់ថា$2^{3^{4n+1}}+3 \vdots 11, \forall n \in{\N}$។
\end{exercise}
\begin{proof}[\solution]
ស្រាយបញ្ជាក់ថា$2^{3^{4n+1}}+3 \vdots 11, \forall n \in{\N}$\\
ដោយ$gcd(2,11)=1$\\
តាមទ្រឹស្តីបទ{\en Fermat}\\
$2^{10}\equiv1\quad(mod11)$\\
យក$3^{4n+1}$ចែកនឹង$10$\\
$\begin{aligned}
\text{គេបាន}\; 3^{4n+1}&=3^{4n}\times3=81^n\times3\\
&\equiv 3\quad(mod10)
\end{aligned}$\\
$\Rightarrow 3^{4n+1}=10k+3$\\
$\begin{aligned}
\text{គេបាន}\; &2^{3^{4n+1}}+3=2^{10k+3}+3=2^{10k}\times2^3+3\\
&\equiv 11\equiv 0\quad(mod11)
\end{aligned}$\\
\fbox{\so $2^{3^{4n+1}}+3$ចែកដាច់នឹង$11$}
\end{proof}
\begin{exercise}
ស្រាយបញ្ជាក់ថា$9^{9^{9^{9^{9}}}}-9^{9^{9^{9}}} \vdots 10$។
\end{exercise}
\begin{proof}[\solution]
ស្រាយបញ្ជាក់ថា$9^{9^{9^{9^{9}}}}-9^{9^{9^{9}}} \vdots 10$\\
យក$9^{9^{9^{9}}}$ចែកនឹង$4$គេបាន\\
$\begin{aligned}
9^{9^{9^{9}}}&=(8+1)^{9^{9^{9}}}\\
&\equiv 1\quad(mod4)
\end{aligned}$\\
$\Rightarrow 9^{9^{9}}=4k+1$\\
$\begin{aligned}
\text{ដូចគ្នាដែរ}\;9^{9^{9}}&=(8+1)^{9^{9}}\\
&\equiv 1\quad(mod4)
\end{aligned}$\\
$\Rightarrow 9^{9^{9}}=4t+1$\\
$\begin{aligned}
\text{គេបាន}\; &9^{9^{9^{9^{9}}}}-9^{9^{9^{9}}}=9^{4k+1}-9^{4t+1}=9^{4k}\cdot9-9^{4t}\cdot9\\
&\equiv 9-9\equiv 0\quad(mod10)
\end{aligned}$\\
$\Rightarrow9^{9^{9^{9^{9}}}}-9^{9^{9^{9}}}\vdots10$\quad(ពិត)\\
\fbox{\so $9^{9^{9^{9^{9}}}}-9^{9^{9^{9}}}$ចែកដាច់នឹង$10$}
\end{proof}
\begin{exercise}
រកលេខមួយខ្ទង់ចុងក្រោយនៃ$3^{2^{1998}}-2^{9^{1998}}$។
\end{exercise}
\begin{proof}[\solution]
រកលេខមួយខ្ទង់ចុងក្រោយនៃ$3^{2^{1998}}-2^{9^{1998}}$\\
$\star$យក$2^{1998}$ចែកនឹង$4$គេបាន\\
$2^{1998}=\left(2^2\right)^{999}\equiv 0\quad(mod4)\\
\Rightarrow2^{1998}=4k\\
\star\text{យក$2^{1998}$ចែកនឹង$4$គេបាន}\\
9^{1998}=(8+1)^{1998}\equiv 1\quad(mod4)\\
\Rightarrow9^{1998}=4t+1 $\\
$\begin{aligned}
\text{គេបាន}\; &3^{2^{1998}}-2^{9^{1998}}=3^{4k}-2^{4t+1}\\
&\equiv 1-6\times2\equiv -11\equiv -20+9\quad(mod10)\\
&\equiv 9\quad(mod10)
\end{aligned}$\\
\fbox{\so $3^{2^{1998}}-2^{9^{1998}}$មានលេខមួយខ្ទង់ចុងក្រោយគឺ$9$}
\end{proof}
\begin{exercise}
 រកលេខមួយខ្ទងចុងក្រោយនៃផលចែកក្នុងវិធីចែក$3^{2^{1930}}+2^{9^{1945}}-19^{5^{1980}}$នឹង$7$។
\end{exercise}
\begin{proof}[\solution]
រកលេខមួយខ្ទង់ចុងក្រោយនៃផលចែក\\
តាង$A=3^{2^{1930}}+2^{9^{1945}}-19^{5^{1980}}$\\
ដោយ$A\equiv 0\quad(mod7)$\\
$\star$យក$2^{1930}$ចែកនឹង$4$គេបាន\\
$2^{1930}\equiv 0\quad(mod10)\\
\Rightarrow 2^{1930}=4k\quad ,k\in \R$\\
$\star$យក$9^{1945}$ចែកនឹង$4$គេបាន\\
$\begin{aligned}9^{1945}&=(8+1)^{1945}
&\equiv 1\quad(mod4)
\end{aligned}$\\
$\Rightarrow 9^{1945}=4t+1$\\
$\star$យក$5^{1980}$ចែកនឹង$4$គេបាន\\
$5^{1980}=(4+1)^{1980}\equiv 1\quad(mod4)\\
\Rightarrow5^{1980}=4y+1$\\
$\star$យក$A$ចែកនឹង$10$\\
$\begin{aligned}
\text{គេបាន}\; A&=3^{4k}+2^{4t+1}-19^{4y+1}\equiv 1+6\times2-19\quad(mod10)\\
&\equiv -6\equiv -10+4\equiv 4\quad(mod10)
\end{aligned}$\\
$\Rightarrow A=r_o\quad(mod70)$\\
យក$r_o=10x+4\quad ,x=0,1,2,3,4,5,6$\\
$\begin{aligned}
\text{មានតែ}\; x=1\Rightarrow r_o&=14\\
&\equiv 0\quad(mod7)
\end{aligned}$\\
$\Rightarrow A\equiv 14\quad(mod70)$
\;
 \begin{tikzpicture}
 \node [rectangle callout,fill=yellow!70,text=black,callout absolute pointer={(0,1)}
] at (5,1) {រកលេខមួយខ្ទង់ចុងក្រោយយើងចែកវានឹង$7$};
  \end{tikzpicture}\\
\fbox{\so លេខមួយខ្ទង់ចុងក្រោយនៃផលចែកគឺ$2$}
\end{proof}
\begin{exercise}
រកលេខមួយខ្ទង់ចុងក្រោយនៃចំនួន$A=2017^{2018^{2019}}$។
\end{exercise}
\begin{proof}[\solution]
រកលេខមួយខ្ទង់ចុងក្រោយនៃចំនួន$A=2017^{2018^{2019}}$\\
$\begin{aligned}
\text{គេបាន}\; A&=2017^{2018^{2019}}=(2010+7)^{2018^{2019}}\\
&\equiv 7^{2018^{2019}}\quad(mod10)
\end{aligned}$\\
$\star$យក$2018^{2019}$ចែកនឹង$4$\\
$\begin{aligned}
\text{គេបាន}\; 2018^{2019}&=(2016+3)^{2019}\\
&\equiv 2^{2019}\equiv 0\quad(mod4)
\end{aligned}$\\
$\Rightarrow 2018^{2019}=4k$\\
$\begin{aligned}
\text{នាំឱ្យ}\; A&\equiv 7^{2018^{2019}}\equiv 7^{4k}\\
&\equiv 1\quad(mod10)
\end{aligned}$\\
\fbox{\so $2017^{2018^{2019}}$មានលេខមួយខ្ទង់ចុងក្រោយគឺ$1$}
\end{proof}
\begin{exercise}
រកលេខពីរខ្ទង់ចុងក្រោយនៃ$14^{14^{14}}$។
\end{exercise}
%\begin{center}
%\begin{tikzpicture}
%\shade[top color=red, bottom color=white, shading angle={135}]
%[draw=black,fill=red!20,rounded corners=1.2ex,very thick] (1.5,.5) -- ++(0,1) -- ++(1,0.3) --  ++(3,0) -- ++(1,0) -- ++(0,-1.3) -- (1.5,.5) -- cycle;
%\draw[very thick, rounded corners=0.5ex,fill=black!20!blue!20!white,thick]  (2.5,1.8) -- ++(1,0.7) -- ++(1.6,0) -- ++(0.6,-0.7) -- (2.5,1.8);
%\draw[thick]  (4.2,1.8) -- (4.2,2.5);
%\draw[draw=black,fill=gray!50,thick] (2.75,.5) circle (.5);
%\draw[draw=black,fill=gray!50,thick] (5.5,.5) circle (.5);
%\draw[draw=black,fill=gray!80,semithick] (2.75,.5) circle (.4);
%\draw[draw=black,fill=gray!80,semithick] (5.5,.5) circle (.4);
%\draw[red,line width=3mm,semithick] (-.5,0) -- (8,0);
%\end{tikzpicture}
%\end{center}
\begin{proof}[\solution]
រកលេខពីរខ្ទង់ចុងក្រោយនៃ$14^{14^{14}}$\\
$\star$យក$14^{14}$ចែកនឹង$20$\\
$\begin{aligned}
\text{គេបាន}\; 14^{14}&=\left(14^2\right)^{7}=(196)^7\\
&=(200-4)^7\equiv -4^7\equiv -(2)^{14}\quad(mod20)\\
&\equiv -\left(2^6\right)^2\times2^2\equiv -(64)^2\times4\quad(mod20)\\
&\equiv -(60+4)^2\times4\equiv -16\times4\equiv -4\quad(mod20)\\
&\equiv -20+16\equiv 16\quad(mod20)
\end{aligned}$\\
$\Rightarrow 14^{14}=20k+16 $\\
$\star$យក$14^{14^{14}}$ចែកនឹង$100$\\
$\begin{aligned}
\text{គេបាន}\; 14^{14^{14}}&=14^{20k+16}=14^{20k}\cdot14^{16}\\
&\equiv 76\left(14^2\right)^8\equiv 76(200-4)^8\quad(mod100)\\
&\equiv 76\cdot4^8\equiv 76\cdot2^{16}\equiv 76\cdot2^{10}\cdot2^6\quad(mod100)\\
&\equiv 76\cdot24\cdot64\equiv 76\cdot36\quad(mod100)\\
&\equiv 36\quad(mod100)
\end{aligned}$\\
\fbox{\so លេខពីរខ្ទង់ចុងក្រោយនៃ$14^{14^{14}}$គឺ$36$}
\end{proof}
\begin{exercise}
 រកលេខពីរខ្ទង់ចុងក្រោយនៃ$2^{9^{1997}}$។
\end{exercise}
\begin{proof}[\solution]
 រកលេខពីរខ្ទង់ចុងក្រោយនៃ$2^{9^{1997}}$\\
 $\star$យក$9^{1997}$ចែកនឹង$20$\\
 $\begin{aligned}
 \text{គេបាន}\; 9^{1997}&=\left(9^2\right)^{998}\times9=(80+1)^{998}\times9\\
 &\equiv 9\quad(mod20)
 \end{aligned}$\\
 $\Rightarrow 9^{1997}=20k+9$\\
 $\begin{aligned}
 \text{នាំឱ្យ}\; 2^{9^{1097}}&=2^{20k+9}=2^{20k}\times2^9\\
 &\equiv 76\times2^9\equiv 76(500+12)\equiv 76\times12\quad(mod100)\\
 &\equiv (900+12)\equiv 12\quad(mod100)\\
 \end{aligned}$\\
 \fbox{\so លេខពីរខ្ទង់ចុងក្រោយនៃ$2^{9^{1997}}$គឺ$12$}
\end{proof}
\begin{exercise}
ស្រាយបញ្ជាក់ថា$9^{9^{9^{9}}}-9^{9^{9}}\vdots 100$។
\end{exercise}
\begin{proof}[\solution]
ស្រាយបញ្ជាក់ថា$9^{9^{9^{9}}}-9^{9^{9}}\vdots 100$\\
$\begin{aligned}
\text{ដោយ}\; 9^9&=\left(9^2\right)^4\times9=(80+1)^4\times9\\
&\equiv 9\quad(mod20)
\end{aligned}$\\
$\Rightarrow 9^9=20k+9$\\
$\begin{aligned}
\text{នាំឱ្យ}\; 9^{9^{9}}&=9^{20k+9}=9^{20k}\times9^9\\
&\equiv 1\times9^9\equiv (10-1)^{9}\quad(mod100)\\
&\equiv (-1)^8\cdot9\cdot10\cdot1^8+(-1)\cdot1\quad(mod100)
 \;
 \begin{tikzpicture}
 \node [rectangle callout,fill=red!30,text=black,callout absolute pointer={(0,1)}
] at (5,1) {$(a-b)^n\equiv (-1)^{n-1}nab^{n-1}+(-1)^nb^n(moda^2)$};
  \end{tikzpicture}\\
&\equiv 89\quad(mod100)
\end{aligned}$\\
$\Rightarrow 9^{9^{9}}=100t+89$\\
$\begin{aligned}
\Leftrightarrow 9^{9^{9^{9}}}&=9^{100t+89}=9^{20(5k+4)+9}\\
&\equiv 9^9\equiv 89\quad(mod100)
\end{aligned}$\\
$\Rightarrow 9^{9^{9^{9}}}-9^{9^{9}}\equiv 89-89\equiv 0\vdots100$\quad(ពិត)\\
\fbox{\so $9^{9^{9^{9}}}-9^{9^{9}}\vdots100$}
\end{proof}
\begin{exercise}
 រកលេខពីរខ្ទង់ចុងក្រោយនៃ$3^{2^{2008}}$។
\end{exercise}
\begin{proof}[\solution]
 រកលេខពីរខ្ទង់ចុងក្រោយនៃ$3^{2^{2008}}$\\
 $\star$យក$2^{2008}$ចែកនឹង$20$\\
 $\begin{aligned}
\text{គេបាន}\; 2^{2008}&=\left(2^2\right)^{1004}=4^{1003}\times4=4(5-1)^{1004}\equiv -4\quad(mod20)\\
&\equiv -20+16\equiv 16\quad(mod20) 
 \end{aligned}$\\
 $\Rightarrow 2^{2008}=20x+16$\\
 $\begin{aligned}
 \text{គេបាន}\; 3^{2^{2008}}&=3^{20x+16}\equiv 1\times3^{36}\quad(mod100)\\
 &\equiv \left(3^4\right)^4\equiv (100-19)^4\equiv 19^4\quad(mod100)\\
 &\equiv \left(19^2\right)^2\equiv 61^2\equiv 21\quad(mod100)
 \end{aligned}$\\
 \fbox{\so លេខពីរខ្ទង់ចុងក្រោយនៃ$3^{2^{2008}}$គឺ$21$}
\end{proof}
\begin{exercise}
 ស្រាយបញ្ជាក់ថា$7^{2^{4n+1}}+4^{3^{4n+1}}-65$ចែកដាច់នឹង$100$។
\end{exercise}
\begin{proof}[\solution]
 ស្រាយបញ្ជាក់ថា$7^{2^{4n+1}}+4^{3^{4n+1}}-65$ចែកដាច់នឹង$100$\\
 $\star$យក$2^{4n+1}$ចែកនឹង$20$\\
 $\begin{aligned}
 \text{គេបាន}\; 2^{4n+1}&=\left(2^4\right)^n\cdot2=2(4)^{2n}=2(5-1)^{2n}=2\times4(5-1)^{2n-1}\\
 &\equiv -8\equiv -20+12\equiv 12\quad(mod20)
 \end{aligned}$\\
 $\Rightarrow2^{4n+1}=20k+12$\\
 $\star$យក$3^{3n+1}$ចែកនឹង$20$\\
 $\begin{aligned}
\text{គេបាន}\; 3^{4n+1}&=3^{4n}\cdot3=(80+1)^{n}\cdot3\\
 &\equiv 3\quad(mod20)
 \end{aligned}$\\
 $\Rightarrow 3^{4n+1}=20t+3$\\
 $\begin{aligned}
 \text{នាំឱ្យ}\; &7^{2^{4n+1}}+4^{3^{4n+1}}-65=7^{20k+12}+4^{20t+3}-65\\
 &\equiv 7^{12}+4^{3}\cdot76-65\equiv\left(7^4\right)^3+76\times64-65\quad(mod100)\\
 &\equiv (2400+1)^3+64-65\equiv 1-1\quad(mod100)\\
 &\equiv 0\vdots100\quad\left(\text{ពិត}\right) 
 \end{aligned}$\\
 \fbox{\so ​$7^{2^{4n+1}}+4^{3^{4n+1}}-65\vdots100$}
\end{proof}
\begin{exercise}
រកលេខបីខ្ទង់ចុងក្រោយនៃ
\begin{enumerate}[a]
\begin{multicols}{2}
\item $2^{2008}$
\item $3^{2018}$។
\end{multicols}
\end{enumerate}
\end{exercise}
\begin{proof}[\solution]
រកលេខបីខ្ទង់ចុងក្រោយនៃ
\begin{enumerate}[a]
\item $2^{2008}$\\
$\begin{aligned}
\text{គេបាន}\; 2^{2008}&=2^{20\times100+8}\\
&\equiv 376\times2^{8}\equiv 256\quad(mod10^3)
\end{aligned}$\\
\fbox{\so លេខបីខ្ទង់ចុងក្រោយនៃ$2^{2008}$គឺ$256$}
\item $3^{2018}$\\
$\begin{aligned}
\text{គេបាន}\;&3^{2018}=3^{20\times1000+18}\\
&\equiv 3^{18}\equiv \left(3^6\right)^3\equiv (729)^3\quad(mod1000)\\
&\equiv (1000-271)^3\equiv -271^3\quad(mod1000)\\
&\equiv -271(271)^2\equiv-271\times441\quad(mod1000)\\
&\equiv -511\equiv (-1000+489)\equiv 489\quad(mod1000)
\end{aligned}$\\
\fbox{\so លេខបីខ្ទង់ចុងក្រោយនៃ$3^{2018}$គឺ$489$}
\end{enumerate}
\end{proof}
\begin{exercise}
 រកលេខបីខ្ទង់ចុងក្រោយនៃ
\begin{enumerate}[a]
\begin{multicols}{2}
\item $2^{9^{2001}}$
\item $14^{14^{14}}$
\item $3^{2^{2007}}$
\item $17^{5^{123}}$។
\end{multicols}
\end{enumerate}
\end{exercise}
\begin{proof}[\solution]
 រកលេខបីខ្ទង់ចុងក្រោយនៃ
\begin{enumerate}[a]
\item $2^{9^{2001}}$\\
$\star$យក$9^{2001}$ចែកនឹង$100$\\
$\begin{aligned}
\text{គេបាន}\; 9^{2001}&=9^{2000}\times9=(10-1)^{2000}\times9\\
&\equiv 1\times9\equiv 9\quad(mod100)
\end{aligned}$\\
$\looparrowright9^{2001}=100k+9$\\
$\begin{aligned}
\Rightarrow 2^{9^{2001}}&=2^{100k+9}\equiv 2^9\times376\quad(mod1000)\\
&\equiv 512\times376\equiv 512\quad(mod1000)
\end{aligned}$\\
\fbox{\so លេខបីខ្ទង់ចុងក្រោយនៃ$2^{9^{2001}}$គឺ$512$}
\item$14^{14^{14}}$\\
$\star$យក$14^{14}$ចែកនឹង$100$\\
$\begin{aligned}
\text{គេបាន}\; 14^{14}&=\left(14^2\right)^7=(200-4)^7\\
&\equiv -4^7\equiv -2^{14}\equiv -2^{10}\times2^4\quad(mod100)\\
&\equiv -(1000+24)\times16\equiv -24\times16\quad(mod100)\\
&\equiv -384\equiv -400+16\equiv 16\quad(mod100)
\end{aligned}$\\
$\Rightarrow 14^{14}=100t+16$\\
$\begin{aligned}
\text{គេបាន}\; 14^{14^{14}}&=14^{100t+16}\\
&\equiv 376\times14^6\equiv 376\cdot2^{16}\cdot7^{16}\quad(mod1000)\\
&\equiv 376\cdot2^{10}\cdot2^6\cdot\left(7^4\right)^4\equiv 376\times1024\times64\times(2401)^4\quad(mod1000)\\
&\equiv 376\times24\times64\times401^4\equiv 376\times1536\times801^2\quad(mod1000)\\
&\equiv 376\times536\times601\equiv 376\times136\quad(mod1000)\\
&\equiv 136\quad(mod1000)
\end{aligned}$\\
\fbox{\so លេខបីខ្ទង់ចុងក្រោយនៃ$14^{14^{14}}$គឺ$136$}
\item $3^{2^{2007}}$\\
$\star$យក$2^{2007}$ចែកនឹង$100$\\
$\begin{aligned}
\text{គេបាន}\; 2^{2007}&=2^7\cdot2^{2000}=2^7\left(2^{10}\right)^{200}\\
&=2^7(1000+24)^{200}=(100+28)(1000+24)^{200}\\
&\equiv 28\times24^{200}\equiv 28\left(24^2\right)^{100}\quad(mod100)\\
&\equiv 28(500+66)^{100}\equiv 28\times66^{100}\quad(mod100)\\
&\equiv 28\times4056^{50}\equiv 28\times56^{50}\quad(mod100)\\
&\equiv 28\times36^{25}\equiv 28\times36\times96^{12}\quad(mod100)\\
&\equiv 28\times36(100-4)^{12}\equiv 28\times36\times4^{12}\quad(mod100)\\
&\equiv 8\times56^{3}\equiv 8\times16\equiv 28\quad(mod100)
\end{aligned}$\\
$\Rightarrow2^{2007}=100k+28$\\
$\begin{aligned}
\text{នាំឱ្យ}\; 3^{2^{2007}}&=3^{100k+28}\equiv 3^{28}\equiv \left(3^7\right)^{4}\quad(mod1000)\\
&\equiv 187^{4}\equiv 969^2\equiv 961\quad(mod1000)
\end{aligned}$\\
\fbox{\so លេខបីខ្ទង់ចុងក្រោយនៃ$3^{2^{2007}}$គឺ$961$}
\item $17^{5^{123}}$\\
$\star$យក$5^{123}$ចែកនឹង$100$\\
$\begin{aligned}
\text{គេបាន}\; 5^{123}&=\left(5^3\right)^{41}=(100+25)^{41}\\
&\equiv 25^{41}\equiv 25\times\left(25^2\right)^{20}\quad(mod100)\\
&\equiv 25\times25^{20}\equiv 25^{20+1}\equiv 5^{2\times20+2}\quad(mod100)\\
&\equiv 25\times25\equiv 25\quad(mod100)
\end{aligned}$\\
$\Rightarrow 5^{123}=100t+25$\\
$\begin{aligned}
\text{នាំឱ្យ}\; 7^{5^{123}}&=7^{100t+25}\equiv7^{25}\quad(mod1000)\\
&\equiv 807^5\equiv 807\times807^4\equiv 807\times249\quad(mod1000) \\
&\equiv 943\quad(mod1000)
\end{aligned}$\\
\fbox{\so លេខបីខ្ទង់ចុងក្រោយនៃ$7^{5^{123}}$គឺ$943$}
\end{enumerate}
\end{proof}
\begin{exercise}
រកពីរចំនួនគត់វិជ្ជមាន$a$និង$b$
\begin{multicols}{2}
\begin{enumerate}[a]
\item ដោយស្គាល់$\begin{cases}
	a+b=72\\gcd(a,b)=12
\end{cases}$
\item ដោយស្គាល់ $\begin{cases}
ab=735\\lcm(a,b)=105
\end{cases}$
\end{enumerate}
\end{multicols}
\end{exercise}
\begin{proof}[\solution]
រកពីរចំនួនគត់វិជ្ជមាន$a$និង$b$
\begin{enumerate}[a]
\item ដោយស្គាល់$\begin{cases}
	a+b=72\\gcd(a,b)=12
\end{cases}$\\
យើងមាន$gcd(a,b)=12$\\
$\looparrowright$តាមនិយមន័យ\\$\begin{cases}12\mid a\\12\mid b\end{cases}
 \Rightarrow\begin{cases}a=12a'\\b=12b'\end{cases}$\\
លក្ខខណ្ឌ$gcd(a',b')=1$\\
$\begin{aligned}
\text{គេបាន}\; a+b&=12a'+12b'\\
&=12(a'+b')
\end{aligned}$\\
$72=12(a'+b')\\
a'+b'=\frac{72}{12}\\
a'+b'=6$\\
$\Rightarrow\begin{cases}a'+b'=6\\gcd(a',b')=1\end{cases}\\
\Leftrightarrow (a',b')=(5,1),(1,5)$\\
$\bullet$ករណី$\begin{cases} a'=5\\b'=1\end{cases}\Rightarrow\begin{cases}a=12\times5\\b=12\times1\end{cases}\Rightarrow\begin{cases}a=60\\b=12\end{cases}$\\
$\bullet$ករណី$\begin{cases}a'=1\\b'=5\end{cases}\Rightarrow\begin{cases}a=12\times1\\b=12\times5\end{cases}\Rightarrow\begin{cases}a=12\\b=60\end{cases}$\\
\fbox{\so ពីរចំនួនគត់$a,b$គឺ$(60,12),(12,60)$}
\item ដោយស្គាល់ $\begin{cases}
ab=735\\lcm(a,b)=105
\end{cases}$\\
$
\text{យើងមាន}lcm(a,b)=105\text{ហើយ}
ab=735
$\\
តាមនិយមន័យ$\begin{cases}105=a\times a'\\105=b\times b'\end{cases}$\\
លក្ខខណ្ឌ$gcd(a',b')=1$\\
$\begin{aligned}
\text{គេបាន}&105^2=ab\times a'b'\\
&105^2=735\times a'b'\\
&a'b'=\frac{105^2}{735}
\end{aligned}$\\
ដោយ$gcd(a',b')=1\\
\Rightarrow (a',b')=(3,5),(5,3)$\\
+ករណី$(a',b')=(3,5)\\
\Rightarrow \begin{cases}a=\frac{15}{a'}\\b=\frac{105}{b'}\end{cases}\Rightarrow \begin{cases}a=\frac{105}{3}=35\\b=\frac{105}{5}=21\end{cases}$\\
+ករណី$(a',b')=(5,3)\\
\Rightarrow \begin{cases}a=\frac{105}{5}=21\\
b=\frac{105}{3}=35\end{cases}$\\
\fbox{\so​ $(a,b)=(35,21),(21,35)$}
\end{enumerate}
\end{proof}
\begin{exercise}
រក$n\in{\N}, n<100$ដែល$gcd(n,252)=7$។
\end{exercise}
\begin{proof}[\solution]
រក$n\in \N,n<100$\\
យើងមាន$gcd(n,252)=7$\\
តាមនិយមន័យ$\begin{cases}
n=7\times n'\\
252=7\times36
\end{cases}$\\
ដែល$gcd(n',36)=1$\\
$
\text{ដោយ} n<100\\
7\times n'<100\\
n'<\frac{100}{14}\\
n<14$\\
$\Rightarrow n'=1,5,7,11,13$\\
$\Leftrightarrow n=7,35,49,77,91$\\
\fbox{\so $n=7,35,49,77,91$}
\end{proof}
\begin{exercise}
ដោយដឹងថា$gcd(a,b)=1$រក$gcd(11a+2b,18a+5b)$។
\end{exercise}
\begin{proof}[\solution]
រក$gcd(11a+2b,18a+5b)$\\
យើងមាន$gcd(a,b)=1$\\
តាង$gcd(11a+2b,18a+5b)=d$\\
តាមនិយមន័យ$\begin{cases}
d\mid 11a+2b\\
d\mid 18a+5b
\end{cases}$\\
$\Rightarrow \begin{cases}
d\mid 18(11a+2b)=18\times11a+36b\\
d\mid 11(18a+5b)=11\times18+55b
\end{cases}$\\
$\Rightarrow d\mid(11\times18a+55b)-(11\times18a+36)\\
\Leftrightarrow d\mid 19b\quad(1)$\\
ម្យ៉ាងទៀត$\begin{cases}
d\mid 55a+10b\\
d\mid 36a+10b
\end{cases}$\\
$\Rightarrow d\mid 55a+10b-(36a+10b)\\
\Leftrightarrow d\mid 19a\quad(2)$\\
តាម$(1)$និង$(2)$គេបាន\\
$d\mid gcd(19a,19b)\\
d\mid 19gcd(a,b)\\
d\mid 19\\
\Rightarrow d=1,19$\\
\fbox{\so $gcd(11a+2b,18a+5b)=19$}
\end{proof}
\begin{exercise}
រកចំនួនគត់ធម្មជាតិ$a$និង$b$ដោយដឹងថា
\begin{enumerate}[a]
\item $a+b=432,gcd(a,b)=36$
\item $a\times b=216,gcd(a,b)=6$
\item $ab=288,gcd(a,b)=6$។
\end{enumerate}
\end{exercise}
\begin{proof}[\solution]
រកចំនួនគត់ធម្មជាតិ$a$និង$b$ដោយដឹងថា
\begin{enumerate}[a]
\item $a+b=432,gcd(a,b)=36$\\
យើងមាន$gcd(a,b)=36\\
a+b=432$\\
តាមនិយមន័យ​ $\begin{cases}
a=36a'\\
b=36b'
\end{cases}\text{ដែល} gcd(a',b')=1\\
a+b=36(a'+b')\\
432=36(a'+b')\\
a'+b'=\frac{432}{36}=12\\
\text{តែ}gcd(a',b')=1\\
\Rightarrow (a',b')=(1,11),(5,7),(11,1),(7,5)\\
\text{គេបាន}(a,b)=(36a',16b')\\
\Rightarrow (a,b)=(36,396),(180,252),(396,36),(252,180)$\\
\fbox{\so $(a,b)=(36,396),(180,252),(396,36),(252,180)$}
\item $a\times b=216,gcd(a,b)=6$\\
យើងមាន$gcd(a,b)=6$\\
$\begin{cases}
a=6\times a'\\
b=6\times b'
\end{cases}\text{លក្ខខ័ណ្ឌ}gcd(a',b')=1\\
ab=36a'b'\\
a'b'=\frac{216}{36}=6$\\
ដោយ​$gcd(a',b')=1\\
\Rightarrow (a',b')=(1,6),(6,1),(2,3),(3,2)\\
\Leftrightarrow (a,b)=(6a',6b')=(6,36),(36,6),(12,18),(18,12)$\\
\fbox{\so $(a,b)=(6,36),(36,6),(12,18),(18,12)$}
\item $ab=288,gcd(a,b)=6$\\
យើងមាន$gcd(a,b)=1\\
\begin{cases}
a=6a'\\
b=6b'
\end{cases}\text{ដែល​}gcd(a',b')=1\\
ab=36a'b'\\
288=36a'b'\\
a'b'=\frac{288}{36}=8\\
\text{ដោយ}gcd(a',b')=1\\
\Rightarrow (a',b')=(1,8),(8,1)\\
\Leftrightarrow (a,b)=(6a',6b')=(6,48),(48,6)$\\
\fbox{\so  $(a,b)=(6a',6b')=(6,48),(48,6)$}
\end{enumerate}
\end{proof}
\begin{exercise}
គេឱ្យប្រភាគសម្រួលមិនបាន$\dfrac{a}{b}$។ ស្រាយថាប្រភាគខាងក្រោមក៏សម្រួលមិនបានដែរ
\begin{multicols}{3}
\begin{enumerate}[a]
\item $\dfrac{a-b}{ab}$
\item $\dfrac{ab}{a^{2}+b^{2}}$
\item $\dfrac{2a+b}{a(a+b)}$ ។
\end{enumerate}
\end{multicols}
\end{exercise}
\begin{proof}[\solution]
ស្រាយថាប្រភាគខាងក្រោមជាប្រភាគសម្រួលមិនបាន
\begin{enumerate}[a]
\item $\dfrac{a-b}{ab}$\\
តាង$gcd(a-b,ab)=d$\\
តាមនិយមន័យ$\begin{cases}
d\mid a-b\\
d\mid ab
\end{cases}\\
\Rightarrow \begin{cases}
d\mid ab-b^2\\
d\mid ab
\end{cases}\Leftrightarrow d\mid b^2\quad(1)$\\
តែ$\begin{cases}
d\mid a^2-ab\\
d\mid ab
\end{cases}\Rightarrow d\mid a^2\quad(2)$\\
តាម$(1)$និង$(2)$គេបាន$d\mid gcd(a^2,b^2)$\\
ដោយ$\frac{a}{b}$ជាប្រភាគសម្រួលមិនបាន$gcd(a,b)=1$\\
$\Rightarrow gcd(a,b)=1\\
\Rightarrow d\mid 1\\
\Rightarrow d=1\\
\Rightarrow gcd(a-b,ab)=1$\\
\fbox{\so ប្រភាគ$\frac{ a-b}{ab}$ជាប្រភាគសម្រួលមិនបាន}
\item $\dfrac{ab}{a^{2}+b^{2}}$\\
តាង$gcd(ab,a^2+b^2)=d$\\
គេបាន
$\begin{cases}
d\mid ab\\
d\mid a^2+b^2
\end{cases}\Rightarrow 
\begin{cases}
d\mid a^2b\\
d\mid a^2b+b^3
\end{cases}\Rightarrow d\mid b^3\quad(1)$\\
ម្យ៉ាងទៀត$\begin{cases}
d\mid ab^2\\
d\mid a^3+ab^2
\end{cases}\Rightarrow d\mid a^3\quad(2)$\\
តាម$(1)$និង$(2)\; d\mid gcd(a^3,b^3)$\\
ដោយ$\frac{a}{b}$ជាប្រភាគសម្រួលមិនបាន\\
$gcd(a,b)=1\\
\Rightarrow gcd(a^3,b^3)=1\\
\Rightarrow d\mid 1\\
\Rightarrow d=1\\
\Rightarrow gcd(ab,a^2+b^2)=1$\\
\fbox{\so ប្រភាគ$\frac{ab}{a^2+b^2}$ជាប្រភាគសម្រួលមិនបាន}
\item $\dfrac{2a+b}{a(a+b)}$\\
តាង$gcd(2a+b,a^2+ab)=d$\\
តាមនិយមន័យ$\begin{cases}
d\mid 2a+b\\
d\mid a^2+ab
\end{cases}\\
\Rightarrow \begin{cases}
d\mid 2a^2+ab\\
d\mid 2a^2+2ab
\end{cases}\\
\Rightarrow d\mid 2a^2+ab-2a^2-2ab\\
\Leftrightarrow d\mid ab\quad(1)$\\
ម្យ៉ាងទៀត​$\begin{cases}
d\mid 2a^2+ab\\
d\mid a^2+ab
\end{cases}\\
\Rightarrow d\mid 2a^2+ab-a^2-ab\\
\Rightarrow d\mid a^2\\
\Leftrightarrow d\mid a\quad(2)$\\
តាម$(1)$និង$(2)\; d\mid b\quad(3)$\\
តាម$(2)$និង$(3)\; gcd(a,b)=1\\
d\mid 1\\
d=1\\
\Rightarrow gcd(2a+b,a^2+ab)=1$\\
\fbox{\so ប្រភាគ$\frac{ 2a+b}{a(a+b)}$ជាប្រភាគសម្រួលមិនបាន}
\end{enumerate}
\end{proof}
\begin{exercise}
 ស្រាយបញ្ជាក់ថាប្រភាគខាងក្រោមសម្រួលមិនបានចំពោះគ្រប់ចំនួនគត់$n$
\begin{multicols}{3}
\begin{enumerate}[a]
\item $\dfrac{12n+1}{30n+2}$
\item $\dfrac{15n^{2}+8n+6}{30n^{2}+21n+13}$
\item $\dfrac{n^{3}+2n}{n^{4}+3n^{2}+1}$ ។
\end{enumerate}
\end{multicols}
\end{exercise}
\begin{proof}[\solution]
ស្រាយបញ្ជាក់ថាប្រភាគខាងក្រោមសម្រួលមិនបានចំពោះគ្រប់ចំនួនគត់$n$
\begin{enumerate}[a]
\item  $\dfrac{12n+1}{30n+2}$\\
តាង$gcd(30n+2,12n+2)d$\\
យក$30n+2=(12n+1)\times2+6n\\
12n+1=6n\times2+1\\
6n=6n\times1+0\\
\Rightarrow gcd(30n+2,12n)=1$\\
\fbox{\so ប្រភាគ$\frac{12n+1}{30n+2}$ជាប្រភាគសម្រួលមិនបាន}
\item  $\dfrac{15n^{2}+8n+6}{30n^{2}+21n+13}$\\
រក$gcd(30n^2+21n+13,15n^2+8n+6)\\
30n^2+21n+13=(15n^2+8n+6)2+5n+1\\
15n^2+8n+6=(5n+1)(3n+1)+5\\
5n+1=5\times n+1\\
5n=5n\times1+0\\
\Rightarrow gcd(30n^2+21n+13,15n^2+8n+6)=1$\\
\fbox{\so ប្រភាគ$\frac{15n^2+8n+6}{30n^2+21n+13}$ជាប្រភាគសម្រួលមិនបាន}
\item $\dfrac{n^{3}+2n}{n^{4}+3n^{2}+1}$\\
រក$gcd(n^4+3n^2+1,n^3+2n)\\
n^4+3n^2+1=(n^3+2n)n+n^2+1\\
n^3+2n=(n^2+1)n+n\\
n=n\times1+0\\
\Rightarrow gcd(n^4+3n^2+1,n^3+2n)=1$\\
\fbox{\so ប្រភាគ$\dfrac{15n^{2}+8n+6}{30n^{2}+21n+13}$ជាប្រភាគសម្រួលមិនបាន}
\end{enumerate}
\end{proof}
\begin{exercise}
ស្រាយបញ្ជាក់ថាចំពោះ$a>1$និង$m>1$គេបាន$gcd\left(\dfrac{a^{m}-1}{a-1},a-1\right)=gcd(m,a-1)$ដែល $a,m\in{\N}$។
\end{exercise}
\begin{proof}[\solution]
ស្រាយបញ្ជាក់ថា$gcd\left(\dfrac{a^{m}-1}{a-1},a-1\right)=gcd(m,a-1)$\\
តាង$d=gcd\left(\frac{a^m-1}{a-1},a-1\right)$\\
តាមនិយមន័យ$\begin{cases}
d\mid \frac{a^m-1}{a-1}\\
d\mid a-1
\end{cases}$\\
$\begin{cases}
d\mid \frac{(a-1)\left(a^{m-1}+a^{m-2}+\cdots+a+1\right)}{a-1}\\
d\mid a-1
\end{cases}\\
\begin{cases}
d\mid \left(a^{m-1}-1\right)+\left(a^{m-2}-1\right)+\cdots+a-1+m\\
d\mid a-1
\end{cases}\\
\begin{cases}
d\mid (a-1)\left(a^{m-2}+a^{m-3}+\cdots+a+1\right)+\cdots+a-1+m\\
d\mid m
\end{cases}\\
\begin{cases}
d\mid m\\
d\mid a-1
\end{cases}\\
\Rightarrow gcd(m,a-1)=d$\quad(ពិត)\\
\fbox{\so $gcd\left(\dfrac{a^{m}-1}{a-1},a-1\right)=gcd(m,a-1)$}
\end{proof}
\begin{exercise}
រកចំនួនគត់ធម្មជាតិ$n$តូចបំផុតដើម្បីឱ្យប្រភាគខាងក្រោមសម្រួលមិនបាន\\$\dfrac{7}{n+9}, \dfrac{8}{n+10}, \dfrac{9}{n+11},\cdots,\dfrac{31}{n+33}$​ ។
\end{exercise}
\begin{proof}[\solution]
រកចំនួនគត់ធម្មជាតិ$n$\\
រាងទូទៅនៃប្រភាគគឺ$\frac{k}{n+k+2},k=7,8,\cdots,31$\\
ប្រភាគ$\frac{k}{n+k+2}$ជាប្រភាគសម្រួលមិនបាន\\
គេបាន$gcd(n+k+2,k)=1\\
\Rightarrow gcd(n+2,k)=1$\\
-បើ$n+2\leqslant6,k=7,8,\cdots,31$\\
គ្មានតម្លៃ$n$ដែល$gcd(n+2,k)=1$\\
-បើ$7\leqslant n+2\leqslant31,k=7,8,\cdots,31$\\
មានតម្លៃ$n+2$ធំជាង$31$គឺ$37$\\
នោះ$n+2=37\\
\Rightarrow n=35$\\
\fbox{\so $n=35$}
\end{proof}
\begin{exercise}
រកចំនួនគត់ធម្មជាត$n$ដើម្បី​ឱ្យ$\dfrac{6n+5}{5n+6}$សម្រួលមិនបាន។
\end{exercise}
\begin{proof}[\solution]
រកចំនួនគត់ធម្មជាតិ$n$\\
រក$gcd(6n+5,5n+6)$\\
តាង$d=gcd(6n+5,5n+6)$\\
តាមនិយមន័យ$\begin{cases}
d\mid 6n+5\\
d\mid 5n+6
\end{cases}$\\
$\begin{cases}
d\mid 30n+25\\
d\mid 30n+36
\end{cases}$\\
$\Rightarrow d\mid (30n+36)-(30n+25)$\\
$d\mid 11$\\
$\Rightarrow d\mid 11\text{ឬ} d\mid 1\Leftrightarrow d=11\text{ឬ}​ d=1$\\
ម្យ៉ាងទៀត$d\mid 6n+5-(5n+6)\\
d\mid n-1\\
\Rightarrow n-1\neq 11k,k\in \Z \\
n\neq 11k+1,k=1,2,\cdots$\\
\fbox{\so $n\neq 11k+1,k=1,2,\cdots$}
\end{proof}
\begin{exercise}
រកគ្រប់ចំនួនគត់$n\in {\N}$ដើម្បីឱ្យ
\begin{enumerate}[a]
\begin{multicols}{3}
\item $n^{4}+4$ជាចំនួនបឋម
\item $n^{1997}+n^{1996}+1$ជាចំនួនបឋម
\item $n^{4}+n^{2}+1$ជាចំនួនបឋម។
\end{multicols}
\end{enumerate}
\end{exercise}
\begin{proof}[\solution]
រកគ្រប់ចំនួនគត់$n\in \N$ដើម្បីឱ្យ
\begin{enumerate}[a]
\item $n^{4}+4$ជាចំនួនបឋម\\
$\begin{aligned}
\text{គេបាន}\; n^4+4&=\left(n^2\right)^2+2^2\\
&=\left(n^2\right)^2+4n^2+2^2-4n^2=\left(n^2+2\right)^2-4n^2\\
&=\left(n^2+2-2n\right)\left(n^2+2+2n\right)=\left(n^2-2n+2\right)\left(n^2+2n+2\right)
\end{aligned}$\\
$n^4+4$ជាចំនួនបឋមកាលណា\\
$n^2-2n+2=1\\
n^2-2n+1=0\\
(n-1)^2=0\\
\Rightarrow n=1$\\
\fbox{\so $n=1$}
\item $n^{1997}+n^{1996}+1$ជាចំនួនបឋម\\
គេបាន$n^{1997}+n^{1996}+1=n^{1997}-n^2+n^{1996}-n+n^2+n+1\\
=n^2\left(n^{1995}-1\right)+n\left(n^{1995}-1\right)+n^2+n+1=\left(n^{1995}-1\right)(n^2+n)+n^2+n+1\\
=\left[\left(n^3\right)^{665}-1\right](n^2+n)+n^2+n+1\\
=\left(n^3-1\right) f(n)(n^2+n)+n^2+n+1
\quad \text{ដែល}f(n)=\left[\left(n^3\right)^{664}+\left(n-1\right)^{663}+\cdots+1\right]\\
=(n-1)(n^2+n+1)f(n)(n^2+n)+n^2+n+1\\
=(n^2+n+1)\left[(n-1)f(n)(n^2+n)+1\right]$\\
ដោយ$n^2+n+1\geq3\; \in \N\\
\Rightarrow (n-1)f(n)(n^2+n)+1=1$\\
ហើយ$n^2+n>0\, f(n)>0\\
n-1=0\\
n=1$\\
\fbox{\so $n=1$}
\item $n^{4}+n^{2}+1$ជាចំនួនបឋម\\
គេបាន$n^4+n^2+1=n^4+2n^2+1-n^2\\
=(n^2+1)^2-n^2=(n^2+1-n)(n^2+1+n)\\
=(n^2-n+1)(n^2+n+1)\\
\Rightarrow n^2-n+1<n^2+n+1\; \in \N$\\
$n^4+n^2+1$ជាចំនួនបឋមកាលណា\\
$n^2-n+1=1\\
n(n-1)=0\\
\Rightarrow n=1$\\
\fbox{\so $n=1$}
\end{enumerate}
\end{proof}
\begin{exercise}
រកចំនួនបឋម$p$ដែល
\begin{enumerate}[a]
\item $2p+1$ជាគូបនៃចំនួគត់ធម្មជាតិ
\item $13p+1$ជាគូបនៃចំនួនគត់ធម្មជាតិ។
\end{enumerate}
\end{exercise}
\begin{proof}[\solution]
រកចំនួនបឋម$p$ដែល
\begin{enumerate}[a]
\item $2p+1$ជាគូបនៃចំនួគត់ធម្មជាតិ\\
$\star$របៀបទី១\\
តាង$2p+1=k^3\; ,k\in \N\\
2p=k^3-1\\
2p=(k-1)(k^2+k+1)$\\
$\bullet$ករណី$\begin{cases}
k-1=2\\
k^2+k+1=p
\end{cases}
\Rightarrow \begin{cases}
k=3\\
3^2+3+1=p
\end{cases}
\Rightarrow \begin{cases}
k=3\\
p=13
\end{cases}$\\
$\bullet \text{ករណី}\begin{cases}
k-1=p\\
k^2+k+1=2
\end{cases}\Rightarrow \begin{cases}
k=p+1\\
k^2+k+1=2
\end{cases}\\
\Rightarrow (p+1)^2+p+1+1-2=0\\
p^2+2p+1+p=0\\
p^2+3p+1=0$គ្មានចំនួនបឋម$p$ណាដែលផ្ទៀងផ្ទាត់នោះទេ\\
\fbox{\so $p=13$}\\
$\star$របៀបទី២\\
តាង$2p+1=k^3$\\
ដោយដឹងថា$2p+1$ជាចំនួនគត់សេស\\
$\begin{aligned}[t]
\text{គេបាន}\;&k^3\text{ជាចំនួនគត់សេស}\\
&k\text{ក៏ជាចំនួនគត់សេសដែរ}
\end{aligned}$\\
តាង$k=2t+1$គេបាន\\
$2p+1=(2t+1)^3\\
2p+1=8t^3+12t^2+6t+1\\
2p=8t^3+12t^2+6t\\
p=4t^3+6t^2+3t\\
p=t(t^2+6t+3)$\\
ដោយ$p$ជាចំនួន​បឋមនិង$t<4t^2+6t+3\\
\Rightarrow t=1\\
\Leftrightarrow p=1(4\times1^2+6\times1+3)\\
\Rightarrow p=13$\\
\fbox{\so $p=13$}
\item $13p+1$ជាគូបនៃចំនួនគត់ធម្មជាតិ\\
តាង$13p+1=t^3\quad t\in\N\\
13p=t^3-1\\
13p=(t-1)(t^2+t+1)$\\
$\bullet \text{ករណី}\begin{cases}
t-1=13\\
t^2+t+1=p
\end{cases}
\Rightarrow \begin{cases}
t=14\\
14^2+14+1=p
\end{cases}\Rightarrow \begin{cases}
t=14\\
p=211 
\end{cases}\\
\bullet \text{ករណី}\begin{cases}
t-1=p\\
t^2+t+1=13
\end{cases}\Rightarrow \begin{cases}
t=p+1\\
t^2+t-12=0
\end{cases}\\
\Rightarrow (p+1)^2+p+1-12=0\\
p^2+2p+1+p-11=0\\
p^2+3p-10=0\\
p^2-2p+5p-10=0\\
p(p-2)+5(p-2)=0\\
(p-2)(p+5)=0$\\
ដោយ$(P+5)>0\\
P-2=0\\
P=2$\\
\fbox{\so $p=2$និង$p=211$}
\end{enumerate}
\end{proof}
\begin{exercise}
ស្រាយបញ្ជាក់ថា$2^{2{10n+1}}+19$និង$2^{3^{4n+1}}+3^{2^{4n+1}}+5$ជាចំនួនសមាសចំពោះ$\forall n \in {\N}$។
\end{exercise}
\begin{proof}[\solution]
ស្រាយបញ្ជាក់ថា$2^{2{10n+1}}+19$និង$2^{3^{4n+1}}+3^{2^{4n+1}}+5$ជាចំនួនសមាស\\
$\star$ស្រាយបញ្ជាក់ថា$2^{2{10n+1}}+19$ជាចំនួនសមាស\\
ដោយ$gcd(2,5)=1\\
2^4\equiv 1\quad(mod5)$\\
យក$2^{10n+1}$ចែកនឹង$4$\\
$\begin{aligned}[t]
2^{10n+1}&=2(2^2)^{5n}=2(4)^{5n}\\
&\equiv 0\quad(mod4)
\end{aligned}$
$\Rightarrow 2^{10n+1}=4k$\\
$\begin{aligned}[t]
\text{គេបាន}\; &2^{2^{10n+1}}+19=\left(2^4\right)^k+19=(15+1)^k+19=(15+1)^k+(20-1)\\
&\equiv 1-1\equiv 0\quad(mod5)
\end{aligned}$\\
$\Rightarrow 2^{2^{10n+1}}+19\vdots 5$\\
\fbox{\so $2^{2^{10n+1}}+19$ជាចំនួនសមាស$\forall n\in \N$}\\
$\star$ស្រាយថា$2^{3^{4n+1}}+3^{2^{4n+1}}+5$ជាចំនួនសមាស\\
ដោយ$\begin{aligned}[t]
&2^{3^{4n+1}}+(2+1)^{2^{4n+1}}+(6-1)\\
&\equiv 1-1\equiv 0\quad(mod2)
\end{aligned}$\\
\fbox{\so $2^{3^{4n+1}}+3^{2^{4n+1}}+5$ជាចំនួនសមាស}
\end{proof}
\begin{exercise}
រកគ្រប់ចំនួនគត់វិជ្ជមាន$a$និង$b$ដើម្បីឱ្យ$a^{4}+4b^{4}$ជាចំនួនបឋម។
\end{exercise}
\begin{proof}[\solution]
រកគ្រប់ចំនួនគត់វិជ្ជមាន$a$និង$b$\\
$\begin{aligned}[t]
\text{គេបាន}\;&a^{4}+4b^{4}=(a^2)^2+4a^2b^2+(2b^2)^2-4a^2b^2=\left(a^2+2b^2\right)^2-4a^2b^2\\
&=\left(a^2+2b^2-2ab\right)\left(a^2+2b^2+2ab\right)
\end{aligned}\\
\forall n\in \N\Rightarrow a^2+2b^2-2ab<a^2+2b^2+2ab$\\
លក្ខខណ្ឌ$a^4+4b^4$ជាចំនួនបឋមគេបាន\\
$a^2+2b^2-2ab=1\\
a^2-2ab+b^2+b^2=1\\
(a-b)^2+b^2=1\\
\forall n\in\N\\
\Rightarrow
\begin{cases}
a-b=0\\
b=1
\end{cases}\Rightarrow \begin{cases}
a=b=1\\
b=1
\end{cases}$\\
\fbox{\so $a=b=1$}
\end{proof}
\begin{exercise}
$p$ជាចំនួនបឋមធំជាង$5$ស្រាយបញ្ជាក់ថា$p^{8n}+3p^{4n}-4 \vdots 5$។
\end{exercise}
\begin{proof}[\solution]
ស្រាយបញ្ជាក់ថា$p^{8n}+3p^{4n}-4 \vdots 5$\\
$\begin{aligned}[t]
\text{តាង}N&=p^{8n}+3p^{4n}-4=p^{8n}+4p^{4n}-p^{4n}-4\\
&=p^{4n}\left(p^{4n}+4\right)-\left(p^{4n}+4\right)=\left(p^{4n}+4\right)\left(p^{4n}-1\right)
\end{aligned}$\\
ដោយ$p$ជាចំនួនបឋមធំជាង$5$\\
តាមទ្រឹស្តីបទ{\en Fermat}\\
$\Rightarrow gcd(p,5)=1\\
p^4\equiv 1\quad(mod 5)\\
p^{4n}\equiv 1\quad(mod5)\\
p^{4n}-1\equiv 0\quad(mod5)
$\\ដោយ$p^{4n}-1\vdots 5$\\
នោះ$\left(p^{4n}+4\right)\left(p^{4n}-1\right)\vdots 5$\quad(ពិត)\\
\fbox{\so $p^{8n}+3p^{4n}-4\vdots 5$}
\end{proof}
\begin{exercise}
រកចំនួនបឋម$p$ដើម្បីឱ្យ$2^{p}+p^{2}$ក៏ជាចំនួនបឋមដែរ។
\end{exercise}
\begin{proof}[\solution]
រកចំនួនបឋម$p$ដើម្បីឱ្យ$2^{p}+p^{2}$ក៏ជាចំនួនបឋមដែរ\\
ឱ្យ$p=2\Rightarrow 2^2+2^2=8$មិនមែនជាចំនួនបឋម\\
បើ$p=3\Rightarrow 2^3+3^2=8+9=17$ជាចំនួនបឋម\\
$p=5\Rightarrow 2^5+5^2=32+25=57$មិនមែនជាចំនួនបឋម\\
ចំពោះ$p>5$មានរាង$6k+1,6k+5$\\
-ចំពោះ$p=6k+1$គេបាន\\
$\begin{aligned}[t]
2^p+p^2&=2^{6k+1}+(6k+1)^2=(3-1)^{6k+1}+(6k+1)^{2}\\
&\equiv -1+1\equiv 0\quad(mod3)
\end{aligned}$\\
-ចំពោះ$p=6k+5$គេបាន\\
$\begin{aligned}[t]
2^p+p^2&=2^{6k+5}+(6k+5)^2=(3-1)^{6k+5}+(6k+5)^2\\
&\equiv 1-1\equiv 0\quad(mod3)
\end{aligned}\\
\Rightarrow p>5\Leftrightarrow 2^p+p^2\vdots 3$នោះ$2^p+p^2$ជាចំនួនសមាស\\
\fbox{\so $p=3, 2^p+p^2$ជាចំនួនបឋម}
\end{proof}
\begin{exercise}
 រកគ្រប់ឬសគត់វិជ្ជមាន $x,y$ ដែលផ្ទៀងផ្ទាត់សមីការៈ
\begin{enumerate}[a]
\begin{multicols}{2}
\item $5x+7y=112$
\item $5x+19y=674$
\item $38x+117y=15$
\item $21x-17y=-3$។
\end{multicols}
\end{enumerate}
\end{exercise}
\begin{proof}[\solution]
 រកគ្រប់ឬសគត់វិជ្ជមាន $x,y$ ដែលផ្ទៀងផ្ទាត់សមីការៈ
 \begin{enumerate}[a]
 \item $5x+7y=112$\\
 យក$x_0=0,y_0=16$
 គេបាន\\
 $-\underline{\begin{cases}
 5x+7y=112\\
 5\times0+7\times 16=112
 \end{cases}}\\
 \begin{aligned}[t]
 \Rightarrow &5x+7(y-16)=0\\
&5x=-7(y-16)\quad(1) \end{aligned}$\\
ដោយ$-7(y-16)\vdots7\\
\Rightarrow 5x\vdots 7\Leftrightarrow x\vdots 7\\
\Rightarrow x=7t\qquad ,t\in\Z$\\
តាម$(1)$គេបាន\\
$5\times7t=-7(y-16)\\
5t=-y+16\\
y=16-5t$\\
ដោយ$x,y\in\N$\\
$\Rightarrow \begin{cases}
x>0\\
y>0
\end{cases}\Rightarrow \begin{cases}
7t>0\\
16-5t>0
\end{cases}\Rightarrow \begin{cases}
t>0\\
t<\frac{16}{5}
\end{cases}\\
0<t<\frac{16}{5}\Rightarrow t\in\{1,2,3\}$\\
នោះគូចម្លើយគឺ$(x,y)=(7t,16-5t)=(7,11),(14,6),(21,1)$\\
\fbox{\so សមីការមានឬសគត់$(7,11),(14,6),(21,1)$}
\item $5x+19y=674$\\
យក$y_0=1\Rightarrow x=\frac{674-19}{5}=131$\\
គេបាន\\
$-\underline{\begin{cases}
5x+19y=674\\
5\times131+19\times1=674
\end{cases}}\\
\begin{aligned}[t]
\Rightarrow& 5(x-131)+19(y-1)=0\\
&5(x-131)=-19(y-1)\quad(1)
\end{aligned}$\\
ដោយ$-19(y-1)\vdots 19\\
\Rightarrow 5(x-131)\vdots 19\\
x-131\vdots19\\
x-131=19t\quad,t\in\Z\\
x=19t+131$\\
តាម$(1)$\\
$5(19t+131-131)=-19(y-1)\\
5\times19t=-19(y-1)\\
5t=-y+1\\
y=1-5t$\\
ដោយ$x,y\in\N\\
\begin{cases}
19t+131>0\\
1-5t>0
\end{cases}\Rightarrow
\begin{cases}
t>-\frac{131}{19}\\
t<\frac{1}{5}
\end{cases}\\
\Rightarrow -\frac{131}{19}<t<\frac{1}{5}$\\
ដោយ$t\in\Z\Rightarrow t\in\{-6,-5,-4,-3,-2,-1,0\}$\\
នោះគូចម្លើយគឺ$(x,y)=(17,31),(36,26),(55,21),(74,16),(93,11),(112,6),(131,1)$\\
\fbox{\so សមីការមានឬសគត់គឺ$(x,y)=(17,31),(36,26),(55,21),(74,16),(93,11),(112,6),(131,1)$}
\item $38x+117y=15$\\
ដោយ$x,y\in\N\\
\Rightarrow 38x+117y\geqslant38+117\\
\Rightarrow 38x+117y=15$គ្មានឬសជាចំនួ​នគត់វិជ្ជមាន\\
\fbox{\so សមីការគ្មានឬសគត់}
\item $21x-17y=-3$\\
យក$x_0=12\Rightarrow y_0=\frac{-3-21\times12}{-17}=15$\\
គេបាន\\$-\underline{\begin{cases}
21x+117y=-3\\
21\times12-17\times15=-3
\end{cases}}\\
\begin{aligned}[t]
\Rightarrow &21(x-12)+17(y-15)=0\\
&21(x-12)=-17(y-15)\quad(1)
\end{aligned}$\\
ដោយ$-17(y-15)\vdots17\\
\Rightarrow 21(x-12)\vdots17\\
\Rightarrow x-12\vdots17\\
x-12=17t\quad,t\in\Z\\
x=17t+12\\
\text{តាម}(1)\\
21(17t+12-12)=17(y-15)\\
21t=y+15\\
y=21t-15\\
\text{ដោយ}x,y\in\N\\
\begin{cases}
17t+12>0\\
21t-15>0
\end{cases}\Rightarrow \begin{cases}
t>-\frac{12}{17}\\
t>\frac{15}{21}
\end{cases}$\\
ដោយ$t\geqslant0\quad,t\in\Z$\\
នោះសមីការ$21x-17y=-3$មានឬសច្រើនរាប់មិនអស់\\
\fbox{\so សមីការមានឬសច្រើនរាប់មិនអស់}
 \end{enumerate}
\end{proof}
\begin{exercise}
\begin{enumerate}[a]
\item នៅលើបន្ទាត់​ $8x-13y+6=0$ ចូររកចំណុចគត់(ជាចំណុចមានកូអរដោនេជាចំនួនគត់)នៅចន្លោះបន្ទាត់ពីរ $x=-10$​ និង $x=5$។
\item ស្រាយបញ្ជាក់ថាក្នុងចតុកោណកែងកំណត់ដោយបន្ទាត់ $x=6, x=42$ និង $y=2, y=17$ មិនមានចំណុចគត់ណានៅលើបន្ទាត់ $3x+5y=7$។
\end{enumerate}
\end{exercise}
\begin{proof}[\solution]

\end{proof}
\newpage
\begin{tikzpicture}[scale=0.15,line cap=round,line join=round,>=triangle 45,x=1.0cm,y=1.0cm]
\draw[->,ultra thick,color=blue] (-39,0.) -- (60,0.);
\foreach \x in {6,42}
\draw[shift={(\x,0)},color=blue] (0pt,2pt) -- (0pt,-2pt) node[below] {\footnotesize $\x$};
\draw[->,ultra thick,color=blue] (0.,-40) -- (0.,40);
\foreach \y in {2,17}
\draw[shift={(0,\y)},color=blue] (2pt,0pt) -- (-2pt,0pt) node[left] {\footnotesize $\y$};
\draw[color=blue] (0pt,-10pt) node[right] {\footnotesize $0$};
\clip(-39.78873239436621,-50.22573363431152) rectangle (60.211267605633815,49.7742663656885);
\draw [line width=1.6pt,color=red] (6.,-40) -- (6.,40);
\draw [line width=1.6pt,color=red] (42.,-40) -- (42.,40);
\draw [line width=1.6pt,color=red,domain=-39:60] plot(\x,{(--2.-0.*\x)/1.});
\draw [line width=1.6pt,color=red,domain=-39:60] plot(\x,{(--17.-0.*\x)/1.});
\draw [line width=1.6pt,color=cyan,domain=-39:60] plot(\x,{(--7.-3.*\x)/5.});
\begin{scriptsize}
\draw[color=cyan] (-30,22) node {\begin{turn}{-30}$3x+5y=7$\end{turn}};
\draw[color=blue] (0,41) node {$y$};
\draw[color=blue] (0,-41) node {$y'$};
\draw[color=blue] (-39,-1.3) node {$x'$};
\draw[color=blue] (59,-1.9) node{$x$};
\end{scriptsize}
\end{tikzpicture}
\begin{ex}{}{fermat}
សចដនន្ថដដចវថដនវ្ថដវនថដ្វដថនវដថនចដសន្
\end{ex}
\end{document}
